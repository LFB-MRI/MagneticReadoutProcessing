

\documentclass [
  english
% 	draft		% falls ohne Bilder gedruckt werden soll (Entwurf)
	]{scrbook}	% KOMA Klasse für Bücher

%%MOD%%
\usepackage{fhacmb}	% Style-File für Titelblatt. Ggf. bei Enning holen

\KOMAoptions{
	parskip=true,		% Absätze mit Abstand
	fontsize=12pt,		% Standardschriftgröße
	toc=flat,		% Inhaltsverzeichnis ohne Einzüge
	twoside=false,		% Einseitig setzen
	numbers=nodotatend,	% Nummerierungen nicht mit Punkt abschließen
% die folgenden Optionen nehmen die entsprechende Dinge ins Inhaltsverzeichnis auf
% mit der bei texlive vorhandenen aktuellen Version von pdflatex funkioniert es nicht mehr
% (bekannter Bug)
	toc=bibliography,	% Literaturverzeichnis ins Inhaltsverzeichnis
	toc=listof,		% Abbildung- und Tabellenverzeichnis ins Inhaltsverzeichnis
	toc=index,		% Stichwortverzeichnis ins Inhaltsverzeichnis
	}
%

\usepackage{amsmath,amssymb}
\numberwithin{equation}{section}

\usepackage{tocloft}    % for list of equations
\usepackage{ragged2e}   % to undo \centering
\usepackage{hyperref}   % to make references hyperlinks
\usepackage{xspace}
% define list of equations
\newcommand{\listequationsname}{\Large{}}
\newlistof{myequations}{equ}{\listequationsname}
\newcommand{\myequations}[1]{
   \addcontentsline{equ}{myequations}{\protect\numberline{\theequation}#1}
}
\setlength{\cftmyequationsnumwidth}{2.3em}
\setlength{\cftmyequationsindent}{1.5em}

% command to box, label, reference, and 
% include noteworthy equation in list of equations
\newcommand{\noteworthy}[2]{
\begin{align} \label{#2} \ensuremath{#1} \end{align} 
\myequations{#2} \centering \small \textit{\xspace \xspace \xspace} \normalsize \justify }


\graphicspath{
    {./}
    {./generated_images/}
}




\IfFileExists{xcolor.sty}{
    \RequirePackage{xcolor}
}{
    \RequirePackage{color}
}


\providecommand{\tightlist}{%
  \setlength{\itemsep}{0pt}\setlength{\parskip}{0pt}}



\usepackage{footnotebackref}
\usepackage{graphicx}




%
% for the background color of the title page
%

%
% break urls
%
\PassOptionsToPackage{hyphens}{url}

%
% When using babel or polyglossia with biblatex, loading csquotes is recommended
% to ensure that quoted texts are typeset according to the rules of your main language.
%
\usepackage{csquotes}
%
% variables for title, author and date
%
\usepackage{titling}
%
% ADD SYNTAX HIGHLIGHTING
%
\input{syntax_highlight}




%%%%%% Immer benötigte Packages
%
\usepackage[T1]{fontenc}		% sonst funktioniert die Silbentrennung bei Umlauten nicht
\usepackage[utf8]{inputenc}	% Eingabedekodierung. Ermöglicht Umlaute. Achtung: Unbedingt mit Betreuer
				% Verwendung der Umlaute-Eingabemethode absprechen. Im Zweifel \"O für Ö
\usepackage[ngerman]{babel}	% Silbentrennung und Sprachanpassung
\usepackage{blindtext}		% Blindtext

%\usepackage[hidelinks]{hyperref}		% Sprungmarken, z.B. im Inhaltsverzeichnis auf Textpassagen



\usepackage{graphicx}		% Definiert o.a. \includegraphics
\usepackage[export]{adjustbox}
\let\oldincludegraphics\includegraphics
\renewcommand{\includegraphics}[2][]{%
  \oldincludegraphics[#1,max width=\linewidth]{#2}}



\usepackage{textcomp}		% Sonst funktioniert z.B. \texteuro nicht
\usepackage{scrlayer-scrpage}	% Package zum Definieren der Kopf- und Fußzeilen
\usepackage{amsmath,amssymb}		% Muss sein
\usepackage{mathrsfs} 	% Weitere Mathematik-Symbole, z.B. Laplace-L
%
%%%%% Anpassung an Formatvorlagen des Fachbereichs
%
\usepackage{helvet}		% Serifenlose Schrift ähnlich Arial
\renewcommand{\familydefault}{\sfdefault}	% als Standardschrift serifenlose Schrift verwenden
%
\usepackage{geometry} 		% Ränder direkt einstellen
\geometry{a4paper, top=20mm, left=30mm, right=20mm, bottom=25mm} % nach Vorgabe
\linespread{1.25} 		% Zeilenabstand nach Vorgabe
%
\usepackage{chngcntr}		% Ändert Verhalten von Countern
\counterwithout{figure}{section}	% Figure-Nummerierung nicht bei section-Wechsel zurücksetzen
\renewcommand{\thefigure}{\textbf\thechapter-\arabic{figure}}	% im Stil 3-2
%
%%%% für das Erzeugen von Grafiken mit Zeichenbefehlen
%
\usepackage{tikz}		% Grundpaket
\usetikzlibrary{shapes,arrows}	% einige Symbolpackages
\usepackage{tikz-cd}		% einige Symbolpackages

%
%%%% TABELLEN
%
\usepackage{colortbl}		% für die Hintergrundfarbe von Tabellen
\usepackage{paralist}		% Weitere Nummeriungsoptionen, z.b. alphabetisch für enumerate/itemize

\usepackage{longtable}% FOR PANDOC TABLE GENERATOR
\usepackage{booktabs} % FOR TOPRULE MIDRULE

%
%%%% CODE
%
\usepackage{listings}

\lstset{ 
%  backgroundcolor=\color{white},   % choose the background color; you must add \usepackage{color} or \usepackage{xcolor}; should come as last argument
  basicstyle=\footnotesize,        % the size of the fonts that are used for the code
  breakatwhitespace=false,         % sets if automatic breaks should only happen at whitespace
  breaklines=true,                 % sets automatic line breaking
  captionpos=b,                    % sets the caption-position to bottom
  extendedchars=true,              % lets you use non-ASCII characters; for 8-bits encodings only, does not work with UTF-8
 % frame=single,	                   % adds a frame around the code
  keepspaces=true,                 % keeps spaces in text, useful for keeping indentation of code (possibly needs columns=flexible)
%  keywordstyle=\color{blue},       % keyword style
  numbers=left,                    % where to put the line-numbers; possible values are (none, left, right)
  numbersep=5pt,                   % how far the line-numbers are from the code
%  rulecolor=\color{black},         % if not set, the frame-color may be changed on line-breaks within not-black text (e.g. comments (green here))
  tabsize=2,	                   % sets default tabsize to 2 spaces
  title=\lstname                   % show the filename of files included with \lstinputlisting; also try caption instead of title
}

%
%%%% UNORDERED LISTS
%
\renewcommand{\labelitemii}{$\circ$} % Bullets für itemize-Listen



%
%%%% PAGE AND FIGURE NUMBERING
%
\usepackage{chngcntr}		% Ändert Verhalten von Countern
\counterwithout{figure}{section}	% Figure-Nummerierung nicht bei section-Wechsel zurücksetzen
\renewcommand{\thefigure}{\textbf\thechapter-\arabic{figure}}	% im Stil 3-2

%\usepackage{subfig}		% Unterfigures mit eigenen Bildunterschriften
\usepackage{caption}
\captionsetup[table]{name=Table}
%\setcounter{tocdepth}{3} % SHOW SUB SUB SUB SUB SECTIONS


%
%%%% acronyms
%



\usepackage{pdflscape}

%\usepackage{hyperref}
%\usepackage[colorlinks=true, urlcolor=blue, linkcolor=red]{hyperref}


\usepackage[acronym]{glossaries}
\makeglossaries

\makeglossaries
\newacronym{ic}{IC}{Integrated Circuit}
\newacronym{mrp}{MRP}{MagneticReadoutProcessing}
\newacronym{mri}{MRI}{Magnetic Resonance Imaging}
\newacronym{rest}{REST}{Representational State Transfer}
\newacronym{pc}{PC}{Personal Computer}
\newacronym{ip}{IP}{Internet Protocol}
\newacronym{usb}{USB}{Universal Serial Bus}
\newacronym{ptp}{PTP}{Precision Time Protocol}
\newacronym{pps}{PPS}{Puls Per Second}
\newacronym{i2c}{I2C}{Inter-Integrated Circuit}
\newacronym{cli}{CLI}{Command Line Interface}
\newacronym{pip}{PIP}{Python Package Installer}
\newacronym{hal}{HAL}{Hardware Abstraction Layer}
\newacronym{pdf}{PDF}{Portable Document Format}
\newacronym{html}{HTML}{Hypertext Markup Language}
\newacronym{ide}{IDE}{Integrated Development Environment}
\newacronym{gui}{GUI}{Graphical User Interface}
\newacronym{sbc}{SBC}{Single Board Computer}
\newacronym{gpio}{GPIO}{General Purpose Input/Output}
\newacronym{uart}{UART}{Universal Asynchronous Receiver / Transmitter}
\newacronym{lut}{LUT}{Lookup Table}
\newacronym{cdc}{CDC}{Communication Device Class}
\newacronym{cad}{CAD}{Computer Aided Design}
\newacronym{json}{JSON}{JavaScript Object Notation}
\newacronym{id}{ID}{Identification Number}
\newacronym{pcb}{PCB}{Printed Circuit Board}
\newacronym{os}{OS}{Operating System}
\newacronym{uuid}{UUID}{Universal Unique Identifier}
\newacronym{nmr}{NMR}{Nuclear Magnetic Resonance}
\newacronym{dut}{DUT}{Device Under Test}
\newacronym{ppm}{ppm}{Parts Per Million}
\newacronym{snr}{SNR}{Signal to Noise Ratio}
\newacronym{sd}{SD}{Standard Deviation}
\newacronym{yaml}{YAML}{Yet Another Markup Language}
\newacronym{dr}{DR}{Dynamic Range}
\newacronym{mds}{MDS}{Minimum Detectable Signal}
\newacronym{cog}{CoG}{Center Of Gravity}
\newacronym{adc}{ADC}{Analog to Digital Converter}
\newacronym{ai}{AI}{Artificial Intelligence}
\newacronym{pla}{PLA}{Polylactic Acid}





%
%%%%%%%%%%  Angaben für Titelseite %%%%%%%%%%%%%%%%%%%%%%
%
% Angaben für Titelseite
\arbeitstyp{Masterarbeit}
\fachbereich{Fachbereich 5}
\studiengang{Information Systems Engineering}
\vertiefung{Systeme und Management}
\titel{Development of a hardware and software framework for the automated characterization
of permanent magnets
for low-field MRI systemsg}
\autor{Marcel Werner Heinrich Friedrich Ochsendorf}
\matrnr{3120232}
\betreuer{Prof. Dr.-Ing. Thomas Dey}
\cobetreuer{Prof. Dr.-Ing. Volkmar Schulz}
%\extbetreuer{- extbetreuer -}
\datum{08.01.2024}
%\dank{- dank -}





%\hypersetup{draft} %DISBALE HYPERLINKS
% https://charly-lersteau.com/posts/2019-12-26-latex-hyperref-error-pdfendlink/
%\usepackage{etoolbox}

%\makeatletter
%\patchcmd\@combinedblfloats{\box\@outputbox}{\unvbox\@outputbox}{}{%
 %   \errmessage{\noexpand\@combinedblfloats could not be patched}%
%}%
%\makeatother

\usepackage{rotating}

\definecolor{LightCyan}{rgb}{0.88,1,1}
\definecolor{Gray}{gray}{0.9}
\usepackage[first=0,last=9]{lcg}
\newcommand{\ra}{\rand0.\arabic{rand}}


% ADD PAGE PACING FOR BETTER READABLE TABLES
\newcolumntype{g}{>{\columncolor{Gray}}c}
\renewcommand*{\arraystretch}{2} %2.0 IS  THE SPACING

\usepackage[
    backend=bibtex,
    style=alphabetic,
  ]{biblatex}
\addbibresource{thesis_references.bib}
%\usepackage[square,numbers]{natbib}
%\bibliographystyle{ieeetr}
%\newcommand{\customcite}[1]{\citepalias{#1}}
%\bibliographystyle{plaindin}
\newcommand{\customcite}[1]{\cite{#1}}

\begin{document}

%\defcitealias{HH21}{HH21}


\defcitealias{GKT16}{GKT16}


\defcitealias{H80}{H80}


\defcitealias{W19}{W19}


\defcitealias{DRW23}{DRW23}


\defcitealias{P17}{P17}


\defcitealias{SB10}{SB10}


\defcitealias{BPA22}{BPA22}


\defcitealias{G17}{G17}


\defcitealias{OTW19}{OTW19}


\defcitealias{NK16}{NK16}


\defcitealias{HMW20}{HMW20}


\defcitealias{WZC21}{WZC21}


\defcitealias{DBN08}{DBN08}


\defcitealias{AFL23}{AFL23}


\defcitealias{OTW21}{OTW21}


\defcitealias{MWO24}{MWO24}


\defcitealias{MMN19}{MMN19}


\defcitealias{CP16}{CP16}


\defcitealias{BC16}{BC16}


\defcitealias{WAH21}{WAH21}


\defcitealias{OB20}{OB20}


\defcitealias{O24}{O24}



%%MOD%%
% Einige Anpassungen müssen nach \begin{document} stehen !!
\renewcaptionname{ngerman}{\figurename}{\textbf Figure} 	% Bild ... statt Abbildung ...
\renewcaptionname{ngerman}{\contentsname}{Contents}% Inhalt statt Inhaltsverzeichnis
\renewcaptionname{ngerman}{\bibname}{Bibliography}
\renewcaptionname{ngerman}{\tablename}{Table}
\renewcaptionname{ngerman}{\listfigurename}{List of Figures}
\renewcaptionname{ngerman}{\listtablename}{List of Tables}
\renewcommand*{\acronymname}{List of Abbreviations}
% printglossaries
%%%%%%%%%%%%%%%%%%%%%%%%%%%%%%%%%%%%%%%%%%%%%%%%%%%%%%%%%%%%
% Titel im FH Style
%%%%%%%%%%%%%%%%%%%%%%%%%%%%%%%%%%%%%%%%%%%%%%%%%%%%%%%%%%%%
\fhacmbtitle{\includegraphics[height=4cm]{fh_logo}}{5pt}{5pt}
%%%%%%%%%%%%%%%%%%%%%%%%%%%%%%%%%%%%%%%%%%%%%%%%%%%%%%%%%%%%
% Erklärung / Geheimhaltung
%%%%%%%%%%%%%%%%%%%%%%%%%%%%%%%%%%%%%%%%%%%%%%%%%%%%%%%%%%%%
%\frontmatter 	% Wenn der Hauptteil mit Seite 1 beginnen soll
\pagestyle{plain}








\frontmatter

%%%%%%%%%%%%%%%%%%%%%%%%%%%%%%%%%%%%%%%%%%%%%%%%%%%%%%%%%%%%
% Erklaerung
%%%%%%%%%%%%%%%%%%%%%%%%%%%%%%%%%%%%%%%%%%%%%%%%%%%%%%%%%%%%
\newpage

\hypertarget{erkluxe4rung}{%
\section*{Erklärung}\label{erkluxe4rung}}
I hereby declare that I have prepared this thesis independently and
without outside assistance. Text passages, which are based literally or
in the sense on publications or lectures of other authors, are marked as
such. The work has not yet been submitted to any other examination
authority and has not yet been published.

Aachen, \datum  \_\_\_\_\_\_\_\_\_\_\_\_,
\_\_\_\_\_\_\_\_\_\_\_\_\_\_\_\_\_\_\_\_\_\_\_\_



%%%%%%%%%%%%%%%%%%%%%%%%%%%%%%%%%%%%%%%%%%%%%%%%%%%%%%%%%%%%
% ABSTRACT
%%%%%%%%%%%%%%%%%%%%%%%%%%%%%%%%%%%%%%%%%%%%%%%%%%%%%%%%%%%%
\newpage

\hypertarget{abstract}{%
\section*{Abstract}\label{abstract}}
Especially in the construction of low-field MRI devices based on
permanent magnets, a large number of magnets are used. In order to
realize a homogeneous B0 field with these magnets, which is necessary
for many setups, the magnetic properties of these magnets should be as
similar as possible.

Due to the complex manufacturing process, especially of neodymium
magnets, the different properties, especially the direction of
magnetization, can deviate from each other, which affects the
homogeneity of the field.

To adjust the field afterwards, a passive shimming process is typically
performed, which is complex and time-consuming and requires manual
corrections to the magnets used.

To avoid this process, magnets can be systematically measured in
advance. However, in this methodology, the recording, data storage and
subsequent evaluation of the data play an important role.

The various existing solutions implement individual aspects, but do not
provide a data pipeline from aqusation to analysis.

For this use case, the MagneticReadoutProcessing library was created,
which implements different aspects of data acquisition, data storage,
analysis, and each intermediate step can be customized by the user
without having to create everything from scratch, favoring an exchange
between different user groups.



\newpage
%%%%%%%%%%%%%%%%%%%%%%%%%%%%%%%%%%%%%%%%%%%%%%%%%%%%%%%%%%%%
% ACRONYM VERZEICHNIS
%%%%%%%%%%%%%%%%%%%%%%%%%%%%%%%%%%%%%%%%%%%%%%%%%%%%%%%%%%%%
\printglossaries

\newpage
%%%%%%%%%%%%%%%%%%%%%%%%%%%%%%%%%%%%%%%%%%%%%%%%%%%%%%%%%%%%
% Inhaltsverzeichnis
%%%%%%%%%%%%%%%%%%%%%%%%%%%%%%%%%%%%%%%%%%%%%%%%%%%%%%%%%%%%
\tableofcontents


%%%%%%%%%%%%%%%%%%%%%%%%%%%%%%%%%%%%%%%%%%%%%%%%%%%%%%%%%%%%
% CONTENT
%%%%%%%%%%%%%%%%%%%%%%%%%%%%%%%%%%%%%%%%%%%%%%%%%%%%%%%%%%%%
\mainmatter	% Wenn der Hauptteil mit Seite 1 beginnen soll
\pagestyle{scrheadings}
%%%%%%%%%%%%%%% Anpassung des Seitenstils an FH-Layoutvorschrift %%%%%%%%%%%%
\renewcommand{\chaptermark}[1]{\markboth{\thechapter\hspace{1cm}#1}{}}	% Kapitel für Headerzeile neu definieren (ohne Nummer)
\chead{}		% Header Mitte löschen
\ihead{\leftmark}	% Kapitelbezeichnung links setzen
\renewcommand{\headfont}{\bfseries}	% Bold-Font für Headerzeile verwenden
\setheadsepline{0.5pt}



\hypertarget{introduction}{%
\chapter{Introduction}\label{introduction}}

As the following, the motivation for the development of this framework
is listed. The chapter provides a brief introduction into the problem
domain, delineates the scope and boundaries of the present research,
surveys the current state of the art in low-field \gls{mri} research and
applications, articulates the research question driving this thesis and
delineates the anticipated usecases and benefits that will be explored
and analysed throughout the study.

\hypertarget{background-and-motivation}{%
\section{Background and Motivation}\label{background-and-motivation}}

Magnetic Resonance Imaging \gls{mri} stands as a cornerstone in clinical
diagnostics, utilizing the principles of nuclear magnetic resonance
\gls{nmr} to generate cross-sectional black-and-white images of the
body. This indispensable method plays a central role in contemporary
medicine and research, contributing significantly to saving lives.
Despite its widespread use, traditional \gls{mri} systems often rely on
large, heavy, and expensive magnets to achieve the necessary homogeneity
of the magnetic field for accurate imaging. \cite{Nitz2016}

Various types of magnets are applicated in different \gls{mri} systems.
Permanent magnets generate a steady yet relatively weak magnetic field,
electro magnets are energized by electrical currents, and
superconducting electro magnets produce magnetic fields through electric
induction. Regardless of the type, the primary objective is to create a
homogeneous magnetic field within the \gls{mri}. The higher the
homogeneity, the more accurate the measurements. This uniform magnetic
field aligns the molecules within the body or object, setting the stage
for a second magnetic system to stimulate these molecules for
spin-measurements.

The challenge with conventional homogeneity systems lies in their
substantial size, weight, and cost. Even when targeting smaller areas of
the body, large devices are often necessary.

In response to this, there is a growing interest in developing low-field
\gls{mri} systems that utilize in many cases permanent magnets. These
systems, while offering advantages in energy efficiency and reduced
complexity, face a significant challenge related to the inherent
variability in the strength of permanent magnets. Achieving homogeneity
in the magnetic field is crucial for accurate imaging or comparative
analyses. While typically calculated images have lower resolution due to
lower magnetic field strength, low-field systems facilitate the
comparison of different field behaviours and the identification of all
kinds of irritations. O'Reilly and Teeuwisse and de Gans
\cite{OReilly2019-rn} have already demonstrated low-cost and
small-scale implementations with low-field \gls{mri} in 2021 calculating
images of a head successfully.

Permanent magnets, usually arranged in a circular ``Halbach array''
inside the \gls{mri}, are commonly used in low-field systems. However,
their drawback is the inherent variability in strength, complicating the
achievement of a homogeneous field and requiring precise strength
information for correct magnet ordering and \gls{mri} construction.

Completed \gls{mri} systems pose significant challenges for
retrospective adjustments, particularly when individual magnets impact
the overall homogeneity of the magnetic field. While deviations in
homogeneity can be measured post-assembly, the intricate task of
readjustment is taken into account. It is less cost-intensive and less
complicated to measure the magnets proactively, prior to the
finalization of the \gls{mri} system.

The focus of this thesis is to improve low-frequency \gls{mri}
technology by examining the usability of magnetic field sensors for
characterising permanent magnets used in these systems. The variability
in the strength of permanent magnets leads to significant difficulties
in constructing an \gls{mri} with the necessary precision for homogenous
field generation.

To address this challenge, the thesis proposes the development of a
comprehensive hardware and software framework. The hardware system aims
to selectively measure magnetic fields at different locations or fully
around a permanent magnet using different sensors. Several existing
open-source software solutions implement individual parts, but do not
provide a complete data processing pipeline from acquisition to
analysis, and their data storage formats are not compatible with each
other.

The accompanying open-source software for this thesis is designed not
only to facilitate measurements with different sensors but also to
enable the characterisation of different objects.

The sensor testing process involves three key test procedures for two
digital sensors. Firstly, the background noise for both sensors is
quantified by measuring with the sensors in a constant environment
without any magnets. Secondly, the linearity of the magnetic fields is
measured for all sensors to detect deviations from the estimated ideal
magnetic curve.

At last, the temperature drift is measured by repeating the background
noise test in different temperature environments. Sensor noise should be
less than \emph{50\(\mu\)T} to characterise precisely a magnetic field
in a Halbach-Array of an \gls{mri}with precision greater than
\emph{1000\gls{ppm}}.

This research initiative contributes to the improvement of low-frequency
\gls{mri} systems by enhancing the accuracy of permanent magnet
characterisation. The outcomes of this thesis provides insights into the
selection and evaluation of sensors for future low-field \gls{mri}
research, ultimately contributing to advancements in medical imaging
technologies.

\hypertarget{low-field-mri}{%
\subsection{Low-Field MRI}\label{low-field-mri}}

For modern medical imaging high-field superconducting magnets dominate
most \gls{mri} machines, providing high black-withe image resolution.
However, the substantial costs, space requirements and safety
considerations cause considerable challenges.

\gls{mri} relies on the presence of a robust magnetic field, and over
time, there has been a continual push to enhance the strength of these
magnetic fields. This strength is quantified in units of Tesla {[}T{]},
commonly referred to as the \emph{B0} field in medical contexts, while
physicists use the term magnetic field induction. The \gls{snr} is
proportional to the magnetic \emph{B0} field; growing magnetic field
leads automatically to higher \gls{snr}. The initial \emph{B0} field of
an \gls{mri} aims to be homogeneous, for image acquisition the second
step requires inhomogeneity of the \emph{B0} field to stimulate spin in
the atoms of materials. For high resolution images, the initial
\emph{B0} field aims to be a homogeneous as possible.

Notably, the focus on high-field systems dominated discussions until
around 1991 when the possibility of constructing \gls{mri} machines with
lower magnetic field strengths came up. This marked a shift in exploring
the potential advantages and applications of low-field \gls{mri}
systems. \cite{Nitz2016}

Low-field magnetic resonance imaging (low-field \gls{mri}) is a
\gls{mri} technique that operates at a lower magnetic field strength
compared to conventional high-field \gls{mri} scanners. Typically, the
magnetic field in low-field \gls{mri}-systems measured between
\emph{0.1T} and \emph{0.3T} compared to the usual \emph{1.5T} to
\emph{3T} and above in high-field \gls{mri} scanners
\cite{Hori2021-pt}.

This technology is used in medical imaging as well as in preclinical
research. The main advantage of low-field \gls{mri}s is the improved
imaging of soft material, especially when examining joints and muscles.
It also offers more cost-effective alternatives to high-field \gls{mri}
systems \cite{Hori2021-pt}, cost reduction, a smaller device
footprint, alleviated safety concerns and leading to diminished image
resolution within clinically feasible scan durations.
\cite{Arnold2023-cn}

Low-field \gls{mri} systems are predominantly composed of permanent
magnets. Through the connection of these permanent magnets, a consistent
magnetic field of up to \emph{0.35T} can be generated. However, this
achievement comes at the cost of an average system weight of \emph{14t}.
Despite their cost-effectiveness in production and maintenance,
permanent magnets show drawbacks such as high temperature dependence and
a limited \gls{snr} due to the constrained field strength.
\cite{Nitz2016}

In particular, the advantages of the small design, the fast and simple
image acquisition and the low costs are advantages that will become
increasingly important in the future. However, the use of permanent
magnets and their structure is particularly important in such systems
and needs to be analysed.

\hypertarget{magnet-system}{%
\subsection{Magnet System}\label{magnet-system}}

\begin{figure}
\centering
\includegraphics{./generated_images/border_Example_Hallbach_ring_with_cutouts_for_eight_magnets.png}
\caption{Example Hallbach ring with cutouts for eight magnets
\label{Example_Hallbach_ring_with_cutouts_for_eight_magnets.png}}
\end{figure}

The positioning of permanent magnets holds an important role in
constructing an \gls{mri} and is influencing the homogeneity of the
\emph{B0} magnetic field. Halbach ring magnets
\cite{Halbach1980DesignOP} have become a common design for low field
\gls{mri} and \gls{nmr} systems \cite{cmr.a.20165}.

This positioning has the ability to generate extremely homogeneous
magnetic flux densities, produce virtually no stray fields and is
particularly attractive for larger magnets as their design has the best
flux-to-mass ratio \cite{Wickenbrock_2021}.

A Halbach ring of this type is usually based on a ring with permanent
magnets arranged in a circle. The graphic
\ref{Example_Hallbach_ring_with_cutouts_for_eight_magnets.png} shows an
example \gls{cad} model of such a ring, in which in this case eight
cubic \emph{12x12x12mm} magnets are embedded to generate homogeneous
magnetic flux densities of around \emph{20mT}.

The homogeneity in this configuration depends, among other things, on
the following main aspects:

\begin{itemize}
\item
  \textbf{Material}: The magnetic properties of a material influence the
  generated field strength. Different materials have different magnetic
  susceptibilities.
\item
  \textbf{Magnetization}: The orientation of the magnetic moments in the
  material influences the field strength. A higher magnetization leads
  to a stronger field strength.
\item
  \textbf{Temperature}: Temperature can influence the magnetic
  properties of a material. The magnetization of some materials
  decreases with increasing temperature.
\item
  \textbf{Manufacturing process}: The manufacturing process can
  influence the magnetic properties of the material. This is dependent
  on the purity of the starting materials used and the processing
  methods. There may also be deviations in field strength if different
  magnets from different production batches are compared.
  \cite{inbook}
\end{itemize}

These aspects can also be applied to individual magnets. As a result,
this also complicates the effect on the structure of a Halbach ring
magnet. If these are joined together to form a ring, positioning
tolerances are also added.

Halbach magnetic arrays present a choice for mobile \gls{nmr} due to
their ability to produce highly homogeneous and robust magnetic fields
per unit of magnetic mass, coupled with minimal stray fields. The term
``Halbach Array'' (commonly known as ``magic rings'') denotes a precise
configuration of permanent magnets designed to amplify magnetic flux on
one side while concurrently mitigating or eliminating it on the opposite
side. \cite{inbook}

The Halbach magnetic array appears as a essential element for future
Magnet characterisation. Based on this fact this thesis centres on
assessing the sensors' applicability, the Halbach magnetic array serves
as a tool for later measurements, although its implementation and
further discussion will not be the primary focus.

In order to compensate for inhomogeneities in a finished system, there
are various so-called shimming procedures which further improve
homogeneity after the system has been assembled. This procedure is
explained in the following chapter.

\hypertarget{shimming-procedure}{%
\subsection{Shimming Procedure}\label{shimming-procedure}}

The shimming process is a essential step in magnetic resonance imaging
\gls{mri} to ensure homogeneous magnetic fields for precise imaging.
Shimming corrects irregularities in the static magnetic field that can
be caused by external influences or internal system errors. This process
optimizes field homogeneity, which is essential for high-resolution and
artifact-free images \cite{10.3389/fphy.2021.704566}.

Optimal homogeneity is attained through intricate designs facilitating
active shimming, a technique essential for achieving high-resolution
spectroscopy. Beyond this, simpler combinations and adaptations of
Halbach rings offer versatility, making them suitable for variable field
magnets or magnets that can be effortlessly opened without applying
force. \cite{inbook}

The sources for the shimming process can be hardware and software based.
Hardware shimming involves the use of gradient and radio frequency coils
that are specifically placed to align the magnetic field. Software
shimming, on the other hand, uses algorithms to adjust the control
parameters of the \gls{mri} system and improve homogeneity
\cite{10.3389/fphy.2021.704566}.

In this thesis, reference is made exclusively to the hardware shimming
processes, since this project is to be used in the future to construct a
low-field \gls{mri} field magnet from permanent magnets.

\hypertarget{state-of-the-art} of actively
used \gls{mri} systems are low-field \gls{mri} (\emph{66.6\% 1.5T
systems, 20\% 3.0 T system, 14.4\% low field with \textless0.5T}).
\cite{Nitz2016}

Within the research domain, various implementations have emerged. An
exemplar instance is the work by O'Reilly, Teeuwisse, and Webb, who
introduced a groundbreaking ``three-dimensional \gls{mri} in a
homogeneous \emph{27cm} diameter Bore Halbach Array magnet''
\cite{OReilly2019-rn} in 2019.This innovative setup is subsequently
employed in 2020 to acquire in vivo MR images, showcasing the practical
applications of their pioneering research \cite{OReilly2021-ep}. In
2023, de Vos, Remis and Webb published a summary of the design of a
point-of-care Halbach array low-field \gls{mri} system
\cite{De_Vos2023-pb}.

Within the research domain, various implementations came up. An exemplar
instance is the work by O'Reilly, Teeuwisse, and Webb, who introduced a
groundbreaking ``three-dimensional \gls{mri} in a homogeneous 27cm
diameter Bore Halbach Array magnet'' \cite{OReilly2019-rn} in 2019.
This innovative setup is subsequently employed in 2020 to acquire in
vivo MR images, showcasing the practical applications of their
pioneering research \cite{OReilly2021-ep}. In 2023, de Vos, Remis
and Webb published a summary of the design of an oint-of-care halbach
array low-field \gls{mri} system \cite{De_Vos2023-pb}.

The Halbach magnet incorporated in this system boasts a 27cm diameter, a
\emph{B0} field strength of 50.4mT, and an impressive homogeneity of
2400\gls{ppm} over a 20cm diameter using smaller magnets
(\(12 x 12 x 12 mm^3\)). This exceptional homogeneity enables the
utilization of coil-based gradients for spatial encoding, significantly
enhancing the flexibility of image acquisition.

The Halbach magnet incorporated in this system boasts a \emph{27cm}
diameter, a \emph{B0} field strength of \emph{50.4 mT}, and an
impressive homogeneity of \emph{2400}\gls{ppm} over a \emph{20cm}
diameter using smaller magnets (\emph{12 x 12 x 12 mm\^{}3}). This
exceptional homogeneity enables the utilization of coil-based gradients
for spatial encoding, significantly enhancing the flexibility of image
acquisition.

To further refine the magnet's homogeneity, optimisation techniques are
employed by adjusting the radius of the Halbach ring along the length of
the magnet. The deliberate choice of smaller magnets, as opposed to
other Halbach designs, serves to compensate for inherent manufacturing
imperfections in each individual magnet. This strategic decision not
only mitigates structural demands on the magnet housing in terms of
strength and weight but also augments safety throughout the construction
process.

\textbf{Magnet Characterisation}

The shimming process described describes how, after the field magnet has
been set up, if the homogeneous magnetic flux densities are not
sufficient, these can be adjusted manually by means of adjustments to
the setup or, if the hardware allows it, by software. In addition, the
causes that cause the inhomogeneity, the output permanent magnet, are
also known. As a result, there is the possibility of using the shimming
process or checking the permanent magnets used in advance before they
are used in a Halbach configuration.

In order to measure the magnets individually, there are already
implementations that use different measurement methods to determine the
field strength of individual magnets and individual measuring points are
recorded. This data is then evaluated in separate software
\cite{Wickenbrock_2021}.

There are two ways of using the data from the magnets that were
previously measured:

\begin{itemize}
\tightlist
\item
  By using binning or sorting algorithms to filter for the most similar
  magnets
\item
  Adjustment of the rotation and position within the Halbach
  configuration
\end{itemize}

This form of data processing of previously characterised magnetic data
is currently being implemented experimentally in projects using various
algorithms \cite{Wickenbrock_2021}. Standard sorting algorithms are
used as well as specialized algorithms for optimizing homogeneity by
rotating the individual magnets in a Halbach ring relative to each other
\cite{HalbachMRIDesigner}.

These are each separate projects that implement individual aspects of
data processing, which realize the process of measuring individual
magnets by manual combination. However, there are still compatibility
problems and limitations in the adaptation of hardware and software.

Special algorithms from various projects are used to optimize
homogeneity. The challenge here is to ensure the seamless integration
and compatibility of these algorithms into the overall process. This
should make it possible to create a workflow from the individual magnet
to the finished optimized \gls{cad} model of a halbach ring.

\hypertarget{aim-of-this-thesis}{%
\section{Aim of this Thesis}\label{aim-of-this-thesis}}

The present work aims to provide an efficient and comprehensive solution
for the design of low-field \gls{mri} devices by developing and
implementing a software and hardware framework.

Within the framework of the \emph{DeLoRI} (Dedicated Low-field \gls{mri}
for breast) project, \emph{Fraunhofer MEVIS} in Bremen is actively
engaged in crafting a compact and mobile low-field \gls{mri} unit
customised specifically for screening purposes. As described before,
since 1991 low-field \gls{mri}s have evolved into a growing realm of
research, showcasing significant opportunities within the field of
medical technology. The efforts by \emph{Fraunhofer MEVIS} exemplifies
the ongoing commitment to utilizing the potential of low-field \gls{mri}
for enhanced breast screening applications.

The focus of the ongoing efforts is to improve the homogeneity of
magnets within low-field \gls{mri}systems below \emph{1000}\gls{ppm},
primarily driven by the goal of establishing a compact low-field
\gls{mri} for breast cancer detection.

Beyond the development and prototype construction of low-field \gls{mri}
scanners, the project encompasses electromagnetic simulation of
components within the low-field \gls{mri}system, coupled with machine
learning-driven control and data acquisition. The resultant software
will be instrumental in reconstruction, with a specific focus on
leveraging AI-based methodologies.

This comprehensive effort serves to bolster the prototyping phase of the
low-field \gls{mri}. Diverging from the approach implemented by
O'Reilly, Teeuwisse, and Webb, \emph{DeLoRI} aims to design an open
\gls{mri}, departing from the circular \gls{mri} configuration discussed
in the publication. This innovation is geared towards streamlining
breast examinations, offering enhanced accessibility, and minimizing the
spatial requirements during installation.

In addition \emph{DeLoRI} efforts to achieve heightened accuracy,
striving for precision levels below 1000\gls{ppm}. While O'Reilly,
Teeuwisse, and Webb were able to modify the ring diameter to influence
field homogeneity, the unique goal here is to characterize the magnets
pre-installation, allowing for proactive assessments of homogeneity
characteristics. This approach aims to provide insights into magnetic
field uniformity before the magnets are integrated, thereby streamlining
the optimisation process. Furthermore,

It is important to note that the primary objective of this thesis is not
merely to characterize the magnet itself; rather, the emphasis lies in
the selection and comparison of potential sensors for the
characterisation process.

To achieve this, a versatile hardware setup is in development, designed
to accommodate various sensors for the measurement of magnets or other
objects. Simultaneously, a software interface is being crafted to
universally read data from different sensors and interact seamlessly
with the diverse firmware associated with various Halbach sensors. A key
feature of this system is the ease of sensor interchangeability,
facilitating adaptability and versatility in the characterisation
process.

The scope of the software library is to lay the foundation for the
systematic characterisation of magnets based on permanent magnets. The
library will enable data acquisition, storage and analysis of magnetic
properties, with customisation possible at each step of the process.
Complete documentation, tutorials and tests will enable users to use the
framework efficiently and adapt it to their specific requirements.

The application of the developed framework for the characterisation of
different magnets and the integration of various available magnetic
field sensors serve the practical application and validation of the
developed solution.

Two sensors have been meticulously chosen for inclusion in the study.
The ultimate objective is to assess whether these selected sensors align
with the stringent criteria of achieving an accuracy level of
1000\gls{ppm}. Furthermore, the study seeks to validate whether the
measuring range of these sensors appropriately corresponds to the
required field strength, ensuring their suitability for the intended
application.

\hypertarget{research-question-and-approach}{%
\section{Research Question and
Approach}\label{research-question-and-approach}}

Concerning the \emph{DeLoRI} project, this study exclusively delves into
the realm of permanent magnets employed for creating a homogeneous
\emph{B0} field through Halbach rings. Other systems and spin generation
for measurements are deliberately excluded from consideration but will
be necessary in later stages of the \emph{DeLoRI} project.

Before naming the research focus, it is important to understand the
difference and connection between deviation and resolution of the
system. To measure a deviation of better than \emph{1000}\gls{ppm} in a
\emph{50mT} magnetic field, a resolution that is less than
\emph{1000}\gls{ppm} of \emph{50mT} is needed.

\[\frac{1000}{1000000} \times 50 \, mT = 0.05 \, mT \]

To convert this into microtesla (\(\mu\)T), a multiplication by
\emph{1000} is needed:

\[ 0.05 \, mT \times 1000 = 50 \, \mu T \]

This means that to measure a deviation of better than
\emph{1000}\gls{ppm} in a \(50mT\) magnetic field, a resolution of less
than \emph{50} \(\mu\)T is needed.

The primary objectives of this work revolve around addressing two
pivotal research questions:

\begin{itemize}
\item
  Sensor characterisation: Can the carefully selected sensors
  effectively measure a magnet? Specifically, this involves
  investigating the saturation of the sensors, the linearity of field
  strength concerning distance from the sensor, and, in a subsequent
  phase, exploring temperature dependence.
\item
  Homogeneity Measurement: Can the chosen sensors be adeptly utilized to
  measure the homogeneity of a Halbach ring-based B0 field within the
  stringent limit of less than \emph{1000}\gls{ppm}? The desired outcome
  is a measurement of less than \emph{50} \(\mu\)T at
  \emph{1000}\gls{ppm}, with a specific focus on determining the
  sensors' viability for noise measurements.
\end{itemize}

It is essential to emphasize that the intention is not to characterize
the magnets; the sensors which might be used for characterisation are be
analysed and evaluated.

This work prioritises the development of both hardware and software
customised for sensor utilization and final testing. Physical properties
and considerations take a secondary role in comparison to the
overarching goal of refining the sensor-based methodologies.

\hypertarget{usecases}{%
\section{Usecases}\label{usecases}}

The following section defines possible usecases that the future project
need to cover. These usecases define the setup of the hard- and
software. These illustrate practical situations to understand the
functionality and added value of the implemented solution for the user.

The usecases were defined in the course of project planning and provide
an overview of how the user interacts with the project and what
functionalities can be expected. In the later accomplished evaluation
process \ref{usecase-evaluation}, the defined usecases are also used as
a reference to demonstrate the implemented capabilities of the solution.
This is essential for understanding the needs of the target group and
designing the end result accordingly.

\begin{enumerate}
\def\labelenumi{\arabic{enumi}.}
\tightlist
\item
  \textbf{Ready to use hardware sensor designs}
\end{enumerate}

A universal, easy-to-integrate Hall sensor design allows users to
evaluate the framework quickly and cost-effectively. The pre-built
hardware sensors provide an optimal solution for research, reducing
development time and achieving repeatable measurement results. Once
successfully evaluated, the Hall sensor design should be easily
adaptable to other sensors without the need for major firmware changes.

\begin{enumerate}
\def\labelenumi{\arabic{enumi}.}
\setcounter{enumi}{1}
\tightlist
\item
  \textbf{Taking automatic measurements from sensors}
\end{enumerate}

The purpose of the framework is to enable the automated acquisition of
measurement data from various connected hardware sensors. The user
should be able to configure various measurement series, which should
then be carried out by the framework without further user interaction.

\begin{enumerate}
\def\labelenumi{\arabic{enumi}.}
\setcounter{enumi}{2}
\tightlist
\item
  \textbf{Open storage formats for data export}
\end{enumerate}

The use of open storage formats for the export of data enables an
interoperable data exchange environment. The implementation of
standardized formats improves the portability and long-term availability
of data. This encourages the exchange and further processing of
measurement data in other software tools.

\begin{enumerate}
\def\labelenumi{\arabic{enumi}.}
\setcounter{enumi}{3}
\tightlist
\item
  \textbf{Ready to integrate data analysis functions}
\end{enumerate}

Once it is possible to record and store measured values, it should be
possible for the user to analyse and visualize this data using various
algorithms. The focus here should be on extending the framework with
user-created algorithms.

\begin{enumerate}
\def\labelenumi{\arabic{enumi}.}
\setcounter{enumi}{4}
\tightlist
\item
  \textbf{User programmable data processing pipelines}
\end{enumerate}

User-programmable data processing pipelines enable the flexible design
of data processing sequences as pipeline by users. The framework should
enable users to create their own pipelines with the previously defined
data analysis functions.

\hypertarget{structure}{%
\section{Structure}\label{structure}}

This work is divided into six main chapters, which deal with the
approach, implementation and evaluation. The techniques and concepts
used are explained in detail. Specific examples provide an overview of
the possible use of the developed solution by the user.

Chapter \ref{unified-sensor}. \textbf{Unified Sensor} refers to the
integration of different sensors into a standardised solution. This
enables simple data acquisition and serves as a basic hardware system on
which the subsequent data processing library can be applied.

Chapter \ref{software-readout-framework}. \textbf{Software Readout
Framework} describes the implementation of the data readout framework.
This includes an explanation of the various modules and specific
application examples.

Chapter \ref{usability-improvements}. \textbf{Usability Improvements}
refers to additional activities to improve user-friendliness. This
includes the optimisation of interfaces, interactions and processes to
ensure intuitive and efficient use of the product. This also includes
the documentation of code and the distribution of the source code as a
package to users.

Chapter \ref{usecase-evaluation}. \textbf{Usecase Evaluation} describes
the application of the framework to the previously defined usecases and
thus forms the basis for later evaluation.

Chapter \ref{evaluation}. \textbf{Evaluation} outlines the evaluation
process for permanent magnets using the developed framework. The
research questions posed regarding the suitability of the sensors used
for the characterisation of permanent magnets are examined.

Chapter \ref{conclusion-and-discussion}. \textbf{Conclusion and
Discussion} bringing together essential research components, it
synthesizes study outcomes, discusses implications, and provides
insights for future work. This section ensures closure and aids readers
in grasping the broader context and significance of the research.

Finally, a comprehensive hardware and software framework needs to be
established, which is capable of measuring diverse objects using various
sensors. Additionally, remarks need to be provided regarding the
suitability of the employed sensors for magnetic field measurements.

\hypertarget{unified-sensor}{%
\chapter{Unified Sensor}\label{unified-sensor}}

A defined main objective of this project is the development of a
cost-effective magnetic field sensors interface that is universally
expandable as well. The focus is on mapping different sensors and being
compatible with different magnet types and shapes. This ensures a wide
range of applications in different scenarios.

Another goal is reproducibility to ensure uniform results, which as a
result reduces the susceptibility to errors. Easy communication with
standard \gls{pc} hardware, which offers a variety of common interface
options, maximizes the user-friendliness.

The flexibility to support different sensors and magnets makes the
system versatile and opens the possibility for use in different
applications. A low-cost magnetic field sensors interface will therefore
not only be economically attractive, but also facilitate the integration
of magnetic field sensors in different contexts.

In addition, the low-cost sensor interface will serve as a development
platform for the data evaluation \gls{mrp} library and provide real
measurement data from magnets. In addition, the interface firmware
creates a basis for the development of a data protocol for exchanging
measured values.

This makes it easy to integrate an own measuring devices into the
\gls{mrp} ecosystem at a later date. This is only possible with a
minimal functional hardware and firmware setup and is developed for this
purpose first.

\hypertarget{sensor-selection}{%
\section{Sensor Selection}\label{sensor-selection}}

The selection process for possible magnetic field sensors initially
focussed on the most common and cost-effective ones, especially those
that are already used in smartphones and are therefore widely available.

A key aspect of this selection is the preference for sensors with
digital interfaces to facilitate implementation in the circuit layout
since these kind of sensors are easy to integrate compared to
non-digital sensors, which require specific frameworks. The integration
of integrated temperature sensors represents a significant enhancement
that will later enable precise temperature compensation.

The use of analog sensors is purposefully avoided, though they are
suitable for more precise measurements and extended measuring ranges.
They were excluded because they require more carefully designed circuits
and more complicated energy management.

In the context of the desired goal of developing a cost-efficient and
universally expandable Hall sensor interface, the decision in favour of
digital sensors seems appropriate.

Focussing on the digital \gls{i2c} interface not only facilitates
implementation, but also contributes to overall cost efficiency. At the
same time, the integration of temperature sensors enables precise
measurements under varying environmental conditions. This strategic
choice forms the basis for a flexible, universally applicable Hall
sensor interface that can be seamlessly integrated into various existing
systems.

\begin{longtable}[]{@{}lllll@{}}
\caption{Implemented digital magnetic field sensors
\label{Implemented_digital_magnetic_field_sensors.csv}}\tabularnewline
\toprule
& TLV493D-A1B6 & HMC5883L & MMC5603NJ & AS5510\tabularnewline
\midrule
\endfirsthead
\toprule
& TLV493D-A1B6 & HMC5883L & MMC5603NJ & AS5510\tabularnewline
\midrule
\endhead
Readout-Axis & 3D & 3D & 3D & 1D\tabularnewline
Temperature-Sensor & yes & no & yes & no\tabularnewline
Resolution {[}uT{]} & 98 & 0.2 & 0.007 & 48\tabularnewline
Range {[}mT{]} & ±130.0 & ±0.8 & ±3 & ±50\tabularnewline
Interface & \gls{i2c} & \gls{i2c} & \gls{i2c} & \gls{i2c}\tabularnewline
\bottomrule
\end{longtable}

The table \ref{Implemented_digital_magnetic_field_sensors.csv} shows a
selection of sensors for which hardware and software support has been
implemented. The resolution of the selected sensors covers the expected
range of values required by the various magnets to be tested.

In the Evaluation \ref{evaluation} chapter, basic characterisation
methods are used to evaluate the sensors listed in
\ref{Implemented_digital_magnetic_field_sensors.csv} with regard to
their sensitivity and other parameters. This is done at this point, as
the components of the readout interface that enable interaction with the
sensors are considered first.

\hypertarget{mechanical-structure}{%
\section{Mechanical Structure}\label{mechanical-structure}}

\begin{figure}
\centering
\includegraphics{./generated_images/border_Mechanical_components_for_the_1D_sensor_using_3D_printed_parts.png}
\caption{Mechanical components for the 1D sensor using 3D printed parts
\label{Mechanical_components_for_the_1D_sensor_using_3D_printed_parts.png}}
\end{figure}

The mechanical design of a sensor is kept as simple as possible so that
it can be replicated as easily as possible. The focus is on providing a
stable foundation for the sensor \gls{ic} and an exchangeable holder for
different magnets.

The following figure
\ref{Mechanical_components_for_the_1D_sensor_using_3D_printed_parts.png},
shows a sectional view of the \gls{cad} drawing of the 1D-Single sensor
\ref{d-single-sensor}.

All parts are produced using the 3D printing additive manufacturing
processes. The sensor circuit board is glued underneath the magnet
holder. This is interchangeable, so different distances between sensor
and magnet can be realised.

The exchangeable magnetic holder (shown in green) can be adapted to
different magnets. It can be produced quickly due to the small amount of
parts used. The two recesses lock the magnet holder with the inserted
magnet over the sensor. The specified tolerances allow the magnet to be
inserted into the holder with repeat accuracy and without backlash. This
is important if several magnets have to be measured, where the
positioning over the sensor must always be the same.

\hypertarget{electrical-interface}{%
\section{Electrical Interface}\label{electrical-interface}}

\begin{figure}
\centering
\includegraphics{./generated_images/border_1D_sensor_schematic_and_circuit_board.png}
\caption{1D sensor schematic and circuit board
\label{1D_sensor_schematic_and_circuit_board.png}}
\end{figure}

The electronics consist of the magnetic field sensor and the electrical
interface to connect it to a \gls{pc} in the form of a microcontroller.

The focus is on utilising existing microcontroller development and
evaluation boards, which already integrate all the components required
for basic operation. This not only enabled a time-saving implementation,
but also ensured a cost-efficient realisation.

All the necessary components and their circuitry are recorded on a
\gls{pcb} \ref{1D_sensor_schematic_and_circuit_board.png} and
subsequently manufactured. In addition, footprints are provided for
various sensor \gls{ic} packages. By placing mounting holes on the
\gls{pcb}, it is possible to attach various mechanical mounts on top of
the sensor \gls{ic}s.

Special attention is paid to the provision of an accessible
SYNC-\gls{gpio} connector. This enables subsequent multi-sensor
synchronization and also offers options for later extensions. This
functionality opens up the possibility of synchronising data from
different sensors to achieve precise and coherent measurement results.
Overall, this integrated approach represents an effective solution for
the flexible evaluation of sensors and helps to optimise the development
process.

\hypertarget{firmware}{%
\section{Firmware}\label{firmware}}

\begin{figure}
\centering
\includegraphics{./generated_images/border_Unified_sensor_firmware_simplified_program_structure.png}
\caption{Unified sensor firmware simplified program structure
\label{Unified_sensor_firmware_simplified_program_structure.png}}
\end{figure}

The microcontroller firmware is software that is executed on a
microcontroller in an embedded system. It controls the hardware and
enables the execution of predefined functions. The firmware is used to
process input data, control output devices and performs specific tasks
according to the program code.

It handles communication with sensors, actuators and other peripheral
devices, processing data and making decisions. Firmware is critical to
the functioning of devices.

The firmware is responsible for detecting the possible connected sensors
\ref{Implemented_digital_magnetic_field_sensors.csv} and query
measurements. This measured data can be forwarded to a host \gls{pc} via
a user interface and can then be further processed there.

An important component is that as many common sensors as possible can be
easily connected without having to adapt the firmware. This modularity
is implemented using abstract class design. These are initiated
according to the sensors found at startup. If new hardware is to be
integrated, only the required functions \ref{lst:CustomSensorClass} need
to be implemented.

\begin{lstlisting}[language={C++}, caption={CustomSensor-Class for adding new sensor hardware support}, label=lst:CustomSensorClass]
#ifndef __CustomSensor_h__
#define __CustomSensor_h__
// register your custom sensor in implemented_sensors.h also
class CustomSensor: public baseSensor
{
public:
  CustomSensor();
  ~CustomSensor();
  // implement depending sensor communication interface
  bool begin(TwoWire& _wire_instance); // I2C
  bool begin(HardwareSerial& _serial_instance); // UART
  bool begin(Pin& _pin_instance); // ANALOG or DIGITAL PIN like onewire
  // FUNCTIONS USED BY READOUT LOGIC
  bool is_valid() override;
  String capabilities() override;
  String get_sensor_name() override;
  bool query_sensor() override;
  sensor_result get_result() override;
};
#endif
\end{lstlisting}

The flow chart
\ref{Unified_sensor_firmware_simplified_program_structure.png} shows the
start process and the subsequent main loop for processing the user
commands and sensor results. When the microcontroller is started, the
software checks whether known sensors are connected to \gls{i2c} or
\gls{uart} interfaces.

If any are found (using a dedicated \gls{lut} with sensor address
translation information), the appropriate class instances are created
and these can later be used to read out measurement results.

The next initialisation system is dedicated for multi-sensor
synchronisation \ref{sensor-syncronisation-interface}. The last step in
the setup is to configure communication with the host or connected
\gls{pc}. All implemented microcontroller platforms used
(\emph{Raspberry Pi Pico}, \emph{STM32F4}) have a \gls{usb} slave port.

The used usb descriptor is a \gls{usb} \gls{cdc}. This is used to
emulate a virtual \emph{RS232} communication port using a \gls{usb} port
on a \gls{pc} and usually no additional driver is needed on modern host
systems.

After execution of the setup routine is completed, the system switches
to an infinite loop, which processes several possible actions. One task
is, to react to user commands which can be sent to the system by the
user via the integrated \gls{cli}. All sensors are read out via a timer
interval set in the setup procedure and their values are stored in a
ring buffer. Ring buffer offers efficient data management in limited
memory. Its cyclic structure enables continuous overwriting of older
data, saves memory space and facilitates seamless processing of
real-time data.

Ring buffers are well suited for applications with variable data rates
and minimise the need for complex memory management. The buffer can be
read out by command and the result of the measurement is sent to the
host. Each sensor measurement result is transmitted from the buffer to
the host together with a time stamp and a sequential number. This
ensures that in a multi-sensor setup with several sensors. The
measurements are synchronized \ref{sensor-syncronisation-interface} in
time and are not out of sequence or drift.

\hypertarget{communication-interface}{%
\subsection{Communication Interface}\label{communication-interface}}

\begin{figure}
\centering
\includegraphics{./generated_images/border_Sensors_(+cli).png}
\caption{Sensors \gls{cli} \label{Sensors_(+cli).png}}
\end{figure}

Each sensor that is loaded with the firmware, registers on to the host
\gls{pc} as a serial interface. There are several ways for the user to
interact with the sensor:

\begin{itemize}
\tightlist
\item
  Use with \gls{mrp} \ref{software-readout-framework}-library
\item
  Stand-alone mode via sending commands using built-in \gls{cli}
\end{itemize}

The \gls{cli} mode is a simple text-based interface with which it is
possible to read out current measured values, obtain debug information
and set operating parameters. This allows to quickly determine whether
the hardware is working properly after installation. The \gls{cli}
behaves like terminal programmes, displaying a detailed command
reference \ref{Sensors_(+cli).png} to the user after connecting. The
current measured value can be output using the \emph{readout} command
\ref{Query_sensors_b_value_using_(+cli).png}.

\begin{figure}
\centering
\includegraphics{./generated_images/border_Query_sensors_b_value_using_(+cli).png}
\caption{Query sensors b value using \gls{cli}
\label{Query_sensors_b_value_using_(+cli).png}}
\end{figure}

The other option is to use the \gls{mrp}
\ref{software-readout-framework}-library. The serial interface is also
used here. However, after a connection attempt by the \gls{hal}
\ref{mrphal} module of the \gls{mrp}
\ref{software-readout-framework}-library, the system switches to binary
mode, which is initiated using the \emph{sbm} command. The same commands
are available as for \gls{cli}-based communication, but in a binary
format.

\hypertarget{sensor-syncronisation-interface}{%
\subsection{Sensor Syncronisation
Interface}\label{sensor-syncronisation-interface}}

\begin{figure}
\centering
\includegraphics{./generated_images/border_Multi_sensor_synchronisation_wiring_example.png}
\caption{Multi sensor synchronisation wiring example
\label{Multi_sensor_synchronisation_wiring_example.png}}
\end{figure}

One problem with the use of several sensors on one readout host \gls{pc}
is that the measurements may drift over time. On the one hand, \gls{usb}
latencies can occur. This can occur due to various factors, including
device drivers, data transfer speed and system resources. High-quality
\gls{usb} devices and modern drivers often minimise latencies.
Nevertheless, complex data processing tasks and overloaded \gls{usb}
ports can lead to delays.

\begin{longtable}[]{@{}lllll@{}}
\caption{Measured sensor readout to processing using host software
\label{Measured_sensor_readout_to_processing_using_host_software.csv}}\tabularnewline
\toprule
\begin{minipage}[b]{0.08\columnwidth}\raggedright
Data-Points\strut
\end{minipage} & \begin{minipage}[b]{0.11\columnwidth}\raggedright
Run runtime {[}ms{]}\strut
\end{minipage} & \begin{minipage}[b]{0.35\columnwidth}\raggedright
Average Sensor communication time per reading {[}ms{]}\strut
\end{minipage} & \begin{minipage}[b]{0.22\columnwidth}\raggedright
Communication jitter time {[}ms{]}\strut
\end{minipage} & \begin{minipage}[b]{0.10\columnwidth}\raggedright
Comment\strut
\end{minipage}\tabularnewline
\midrule
\endfirsthead
\toprule
\begin{minipage}[b]{0.08\columnwidth}\raggedright
Data-Points\strut
\end{minipage} & \begin{minipage}[b]{0.11\columnwidth}\raggedright
Run runtime {[}ms{]}\strut
\end{minipage} & \begin{minipage}[b]{0.35\columnwidth}\raggedright
Average Sensor communication time per reading {[}ms{]}\strut
\end{minipage} & \begin{minipage}[b]{0.22\columnwidth}\raggedright
Communication jitter time {[}ms{]}\strut
\end{minipage} & \begin{minipage}[b]{0.10\columnwidth}\raggedright
Comment\strut
\end{minipage}\tabularnewline
\midrule
\endhead
\begin{minipage}[t]{0.08\columnwidth}\raggedright
1\strut
\end{minipage} & \begin{minipage}[t]{0.11\columnwidth}\raggedright
9453\strut
\end{minipage} & \begin{minipage}[t]{0.35\columnwidth}\raggedright
1.44\strut
\end{minipage} & \begin{minipage}[t]{0.22\columnwidth}\raggedright
0\strut
\end{minipage} & \begin{minipage}[t]{0.10\columnwidth}\raggedright
\strut
\end{minipage}\tabularnewline
\begin{minipage}[t]{0.08\columnwidth}\raggedright
1\strut
\end{minipage} & \begin{minipage}[t]{0.11\columnwidth}\raggedright
9864\strut
\end{minipage} & \begin{minipage}[t]{0.35\columnwidth}\raggedright
1.5\strut
\end{minipage} & \begin{minipage}[t]{0.22\columnwidth}\raggedright
0\strut
\end{minipage} & \begin{minipage}[t]{0.10\columnwidth}\raggedright
\strut
\end{minipage}\tabularnewline
\begin{minipage}[t]{0.08\columnwidth}\raggedright
10\strut
\end{minipage} & \begin{minipage}[t]{0.11\columnwidth}\raggedright
12984\strut
\end{minipage} & \begin{minipage}[t]{0.35\columnwidth}\raggedright
1.22\strut
\end{minipage} & \begin{minipage}[t]{0.22\columnwidth}\raggedright
0.9\strut
\end{minipage} & \begin{minipage}[t]{0.10\columnwidth}\raggedright
\strut
\end{minipage}\tabularnewline
\begin{minipage}[t]{0.08\columnwidth}\raggedright
10\strut
\end{minipage} & \begin{minipage}[t]{0.11\columnwidth}\raggedright
12673\strut
\end{minipage} & \begin{minipage}[t]{0.35\columnwidth}\raggedright
1.13\strut
\end{minipage} & \begin{minipage}[t]{0.22\columnwidth}\raggedright
1.1\strut
\end{minipage} & \begin{minipage}[t]{0.10\columnwidth}\raggedright
\strut
\end{minipage}\tabularnewline
\begin{minipage}[t]{0.08\columnwidth}\raggedright
10\strut
\end{minipage} & \begin{minipage}[t]{0.11\columnwidth}\raggedright
43264\strut
\end{minipage} & \begin{minipage}[t]{0.35\columnwidth}\raggedright
2.19\strut
\end{minipage} & \begin{minipage}[t]{0.22\columnwidth}\raggedright
8.2\strut
\end{minipage} & \begin{minipage}[t]{0.10\columnwidth}\raggedright
96\% system load\strut
\end{minipage}\tabularnewline
\bottomrule
\end{longtable}

The table
shows\ref{Measured_sensor_readout_to_processing_using_host_software.csv}
shows various jitter measurements. These were performed on a
\emph{RaspberryPi 4 4GB}-\gls{sbc} together with an \emph{1D: Single
Sensor} \ref{d-single-sensor} and the following software settings:

\begin{itemize}
\tightlist
\item
  \emph{Raspberry Pi OS Lite} - \gls{os} \emph{Debian bookworm x64},
\item
  \gls{mrp} \ref{software-readout-framework}-library - Version
  \emph{1.4.1}
\item
  Unified Sensor \ref{unified-sensor}-firmware - Version \emph{1.0.1}
\end{itemize}

It can be seen that a jitter time of up to an additional \emph{1ms} is
added between the triggering of the measurements by the host system and
the receipt of the command by the sensor hardware. If the host system is
still under load, this value increases many times over. This means that
synchronising several sensors via the \gls{usb} connection alone is not
sufficient.

The other issue is sending the trigger signal from the readout software
\ref{software-readout-framework}. Here too, unpredictable latencies can
occur, depending on which other tasks are also executed on this port.

In order to enable the most stable possible synchronisation between
several sensors, an option has already been created to establish an
electrical connection between sensors. This is used together with the
firmware to synchronise the readout intervals. The schematic
\ref{Multi_sensor_synchronisation_wiring_example.png} shows how several
sensors must be wired together in order to implement this form of
synchronisation.

\begin{figure}
\centering
\includegraphics{./generated_images/border_Unified_sensor_firmware_multi_sensor_synchronisation_procedure.png}
\caption{Unified sensor firmware multi sensor synchronisation procedure
\label{Unified_sensor_firmware_multi_sensor_synchronisation_procedure.png}}
\end{figure}

Once the hardware has been prepared, the task of the firmware of the
various sensors is to find a common synchronisation clock. To do this,
the function \emph{register irq on sync pin} is overwritten. To set one
\emph{primary} and several \emph{secondary} sensors, each sensor waits
for an initial pulse on the SYNC-\gls{gpio}
\ref{Unified_sensor_firmware_multi_sensor_synchronisation_procedure.png}.
Each sensor starts a random timer beforehand, which sends a pulse on the
sync line. All others receive this and switch to \emph{secondary} mode
and synchronise the measurements based on each sync pulse received.

Since the presumed \emph{primary} sensor cannot register its own sync
pulse (because the pin is switched to output), there is a timeout branch
condition \emph{got pulse within 1000ms} and this becomes the
\emph{primary} sensor. This means that in a chain of sensors there is
exactly one \emph{primary} and many \emph{secondary} sensors.

In single-sensor operation, this automatically jumps to \emph{primary}
sensor operation through the \emph{got impulse within 1000ms} branch
result. The synchronisation status can be queried via the user interface
\ref{communication-interface} using the \emph{opmode}
\ref{Query_opmode_using_(+cli).png} command. An important aspect of the
implementation here is that there is no numbering or sequence of the
individual sensors.

This means that for the subsequent readout of the measurements, it is
only important that they are taken at the same interval across all
sensors. The sensor differentiation takes place later in the \gls{mrp}
\ref{software-readout-framework}-library by using the sensor \gls{uuid}.

\begin{figure}
\centering
\includegraphics{./generated_images/border_Query_opmode_using_(+cli).png}
\caption{Query opmode using \gls{cli}
\label{Query_opmode_using_(+cli).png}}
\end{figure}

\hypertarget{example-sensors}{%
\section{Example Sensors}\label{example-sensors}}

Two functional sensor platforms
\ref{Build_sensors_with_different_capabilities.csv} are built in order
to create a solid test platform for later tests and for the development
of the \gls{mrp} \ref{software-readout-framework}-library with the
previously developed sensor concepts.

\begin{longtable}[]{@{}llll@{}}
\caption{Build sensors with different capabilities
\label{Build_sensors_with_different_capabilities.csv}}\tabularnewline
\toprule
& 1D\ref{d-single-sensor} & 1D: dual sensor & 3D:
Fullsphere\ref{d-fullsphere}\tabularnewline
\midrule
\endfirsthead
\toprule
& 1D\ref{d-single-sensor} & 1D: dual sensor & 3D:
Fullsphere\ref{d-fullsphere}\tabularnewline
\midrule
\endhead
Maximal magnet size & Cubic 30x30x30 & Cubic 30x30x30 & Cubic
20x20x20\tabularnewline
Sensor type & MMC5603NJ & TLV493D & TLV493D\tabularnewline
Sensor count & 1 & 2 & 1\tabularnewline
Scanmode & static (1 point) & static (2 points) & dynamic
(fullsphere)\tabularnewline
\bottomrule
\end{longtable}

These cover all the required functions described in the usecases
\ref{usecases}. The most important difference, apart from the sensor
used, is the \emph{scan mode}. In this context, this describes whether
the sensor can measure a \emph{static} fixed point on the magnet or if
the sensor can move \emph{dynamically} around the magnet using a
controllable manipulator.

In the following, the hardware structure of a \emph{static} and
\emph{dynamic} sensor is described. For the \emph{static} sensor, only
the \emph{1D} variant is shown, as this does not differ significantly
from the structure of the \emph{1D: dual sensor}, except it uses two
\emph{TLV493D} sensors, mounted above and on top of the magnet.

\hypertarget{d-single-sensor}{%
\subsection{1D: Single Sensor}\label{d-single-sensor}}

\begin{figure}
\centering
\includegraphics{./generated_images/border_1D_sensor_construction_with_universal_magnet_mount.png}
\caption{1D sensor construction with universal magnet mount
\label{1D_sensor_construction_with_universal_magnet_mount.png}}
\end{figure}

The 1D sensor
\ref{1D_sensor_construction_with_universal_magnet_mount.png} is the
simplest possible sensor that is compatible with the Unified Sensor
firmware \ref{firmware} platform.

The electrical level here is based on a \emph{Raspberry-Pi Pico}
together with the \emph{MMC5603NJ} magnetic sensor. The mechanical setup
consists of four 3D printed components, which are fixed together with
nylon screws to minimise possible influences on the measurement.

Since the \emph{MMC5603NJ} only has limited measurement range of total
\emph{6uT}, even small coin sized neodymium magnets already saturates
the sensor. It is possible to mount 3D printed spacers over the sensor
to increase the distance between the magnet and the sensor and thus also
measure these magnets.

The designed magnet holder can be adapted for different magnet shapes
and can be placed on the spacer without backlash in order to be able to
perform a repeatable measurement without introducing measurement
irregularities by mechanically changing the magnet.

\hypertarget{d-full-sphere}{%
\subsection{3D: Full Sphere}\label{d-full-sphere}}

\begin{figure}
\centering
\includegraphics{./generated_images/border_Full-Sphere_sensor_implementation_using_two_Nema17_stepper_motors_in_a_polar_coordinate_system.png}
\caption{Full-Sphere sensor implementation using two Nema17 stepper
motors in a polar coordinate system
\label{Full-Sphere_sensor_implementation_using_two_Nema17_stepper_motors_in_a_polar_coordinate_system.png}}
\end{figure}

The 3D full sphere sensor
\ref{Full-Sphere_sensor_implementation_using_two_Nema17_stepper_motors_in_a_polar_coordinate_system.png}
offers the possibility to create a 3D map of the inserted magnet.

The graphic
\ref{3D_plot_of_an_N45_12x12x12_magnet_using_the_3D_fullsphere_sensor.png}
shows the visualisation of such a scan in the form of a spherical 3D
map. On the sphere is the magnetic field strength, which is detected by
the sensor at the position. The transition from a fully positive field
strength (red) to a negative field strength (blue) is clearly
recognisable and corresponds to the orientation of the magnet in the
holder.

The magnet sensor is mounted on a movable arm, which can move 180
degrees around the magnet on one axis. In order to be able to map the
full sphere, the magnet is mounted on a turntable. This permits the
manipulator to move a polar coordinate system.

\begin{figure}
\centering
\includegraphics{./generated_images/border_3D_plot_of_an_N45_12x12x12_magnet_using_the_3D_fullsphere_sensor.png}
\caption{3D plot of an N45 12x12x12 magnet using the 3D fullsphere
sensor
\label{3D_plot_of_an_N45_12x12x12_magnet_using_the_3D_fullsphere_sensor.png}}
\end{figure}

As the magnets in the motors, as with the screws used in the 1D sensor,
can influence the measurements of the magnetic field sensor, the
distance between these components and the sensor or magnets is
increased. The turntable and its drive motor are connected to each other
via a belt.

On the electrical side, consists of a \emph{SKR-Pico} stepper motor
controller on the one hand side and a \emph{TLV493D} magnetic field
sensor on the other hand side. This is chosen because of its larger
measuring range and can therefore be used more universally without
having to change the sensor of the arm.

\hypertarget{integration-of-an-industry-teslameter}{%
\subsection{Integration of an
Industry-Teslameter}\label{integration-of-an-industry-teslameter}}

As the sensors shown so far relate exclusively to self-built, low-cost
hardware, the following section shows how existing hardware can be
integrated into the system. A temperature-compensated \emph{Voltcraft
GM-70} telsameter
\ref{Voltcraft_GM70_teslameter_with_custom_(+pc)_interface_board.png} is
used, which has a measuring range of \emph{0T} to \emph{3T} with a
resolution of \emph{0.1mT}. It offers an \emph{RS232} interface with a
documented protocol for connection to a \gls{pc}.

This connectivity makes it possible to make the device compatible with
the unified sensor ecosystem using a separate interface software
\cite{VoltcraftGM70Rest} executable on the host \gls{pc}. However,
it does not offer the range of functions that the unified sensor
firmware offers.

Another option is a custom interface board between the meter and the PC.
This is a good option as many modern \gls{pc}s or \gls{sbc}s no longer
offer an physical \emph{RS232} interface. As with the other sensors,
this interface consists of a \emph{Raspberry-Pi Pico} with an additional
level shifter.

The teslameter is connected to the microcontroller using two free
\gls{gpio}s in \gls{uart} mode. The firmware is adapted using a separate
build configuration. In order to be able to read and correctly interpret
the data from the microcontroller, the serial protocol of the sensor is
implemented in a customised version of the \emph{CustomSensor} class
\ref{lst:CustomSensorClass}.

This software or hardware integration can be carried out on any other
measuring device with a suitable communication interface and a known
protocol thanks to the modular design.

\begin{figure}
\centering
\includegraphics{./generated_images/border_Voltcraft_GM70_teslameter_with_custom_(+pc)_interface_board.png}
\caption{Voltcraft GM70 teslameter with custom \gls{pc} interface board
\label{Voltcraft_GM70_teslameter_with_custom_(+pc)_interface_board.png}}
\end{figure}

\hypertarget{software-readout-framework}{%
\chapter{Software Readout Framework}\label{software-readout-framework}}

The software readout framework is the central software component that is
developed as part of this work. This software framework is intended to
provide a user-oriented data acquisition and analysis environment. For
this purpose, typical individual steps that occur in relation to these
tasks are implemented:

\begin{itemize}
\tightlist
\item
  Data acquisition - from hardware sensors \ref{unified-sensor} or other
  data sources
\item
  Storage - export of data in various open formats
  \ref{storage-and-datamanagement}
\item
  Analysis - algorithms to analyse different data sets \ref{analysis}
\end{itemize}

All these possible task parts is divided into different blocks and users
is given the possibility of adding their own functionalities.

As the following, this concept is referred to as \emph{user interaction
points} \ref{user-interaction-points} and is explained in the following
chapter.

\hypertarget{user-interaction-points}{%
\section{User Interaction Points}\label{user-interaction-points}}

User interaction points represent the core concept of the developed
library and are intended to provide user-friendliness on the one hand
and the rapid development of own analysis and optimisation algorithms on
the other.

For this purpose, the library is divided into individual modules, which
are shown in the graphic
\ref{MRP_library_module_high_level_overview.png}. In combination, these
represent a typical measurement-analysis-evaluation workflow of data.
For this purpose, a module system with standardised functional patterns
and data types is developed and packed together in a extendable Python
library.

\newpage

According to this concept, the user should be able to replace individual
components from this chain with their own modules without having to
worry about implementing other of these to make the project work.

\begin{figure}
\centering
\includegraphics{./generated_images/border_MRP_library_module_high_level_overview.png}
\caption{MRP library module high level overview
\label{MRP_library_module_high_level_overview.png}}
\end{figure}

\hypertarget{user-interaction-points-example}{%
\subsection{User Interaction Points
Example}\label{user-interaction-points-example}}

The following example shows the advantages of using the \emph{User
interaction points}:

A project called \emph{HalbachOptimisation}
\cite{HalbachOptimisation} implements a data analysis step and
optimizes a magnetic field that is as homogeneous as possible within a
circular section using given mechanical dimensions as input parameters
of the magnets used. For this purpose, a mutation of the magnet
positions and rotations is performed. The result is a list of positions
for each magnet.

The \emph{HalbachMRIDesigner} \cite{HalbachMRIDesigner} opensource
project, can generate basic \gls{cad} drawings for \gls{mri} magnets in
a Halbach configuration. To do this, the number of magnets and
additional parameters for the properties of the \gls{cad} model to be
created are passed to the function provided as input parameters using a
\gls{json} file. The result is an \emph{OpenSCAD} \cite{OpenSCAD}
based 3D model of the magnet holder.

As a result, there are two projects which are both suitable for the task
of optimizing and creating Halbach magnets for \gls{mri} applications.
The data structures are not compatible with each other. However, they
are executed manually one after the other to obtain a final result with
manual data conversation.

The library created is intended to solve one problem by providing
standardized and flexible data structures for use with this form of
magnetic field data. By separating the processing pipeline into defined
sub-steps, it should be possible to make individual modules and as a
result functionalities interchangeable by the user.

The implementation of the same functionalities looks as follows after
using the library:

\begin{enumerate}
\def\labelenumi{\arabic{enumi}.}
\tightlist
\item
  \textbf{Create a static set of magnets}
\end{enumerate}

The input parameters of the \emph{HalbachOptimisation}
\cite{HalbachOptimisation} project are on the one hand the
mechanical dimensions and the number of magnets to be used. We assume
ideal magnets here. However, it should also be possible to import field
data in a more measured form later on. Using the DataAquisition
sub-step, it is possible to generate any number of ideal magnets.

\begin{enumerate}
\def\labelenumi{\arabic{enumi}.}
\setcounter{enumi}{1}
\tightlist
\item
  \textbf{Add custom analysis processing step}
\end{enumerate}

Next, the user creates his own analysis step in order to be able to call
up its functions. In the case of the \emph{HalbachOptimisation}
\cite{HalbachOptimisation} project, the function signature of the
start function must be changed. This receives the result of the previous
step, in this case the generated magnet data. By optionally setting meta
data in the universal library data type, constants can be replaced in
the analysis function and made dynamically configurable. The return
result also corresponds to this data type so that subsequent steps are
compatible and contains the magnet data with modified position and
rotation data.

\begin{enumerate}
\def\labelenumi{\arabic{enumi}.}
\setcounter{enumi}{2}
\tightlist
\item
  \textbf{Generate fabrication data}
\end{enumerate}

The last step is to call up the \emph{HalbachMRIDesigner}
\cite{HalbachMRIDesigner} project, which creates the \gls{cad} model
of the magnet holder. The data can also be exported as files here. To
make the project compatible, the function signatures are also adapted
here. In this case, more changes are required, as the configuration file
is loaded from the file system. This logic must be removed in order to
use the added meta data in the input parameter instead.

After these customisation steps, it is possible to execute both projects
one after the other and all required configuration parameters are
contained in the data structure looped through the individual steps as
meta data. This also maps the functionality of a project file, which can
be executed or passed on repeatedly.

This also fulfils the goal of making individual user-created algorithms
interchangeable. If the user now wishes to use a different \gls{cad}
algorithm instead of the \emph{HalbachMRIDesigner}
\cite{HalbachMRIDesigner}, the other steps can simply be preserved
and only the new step needs to be implemented.

\hypertarget{modules}{%
\section{Modules}\label{modules}}

In order to realise the concept of user interaction points, the library
is divided into different modules. These modules can be divided into two
main categories:

\begin{itemize}
\tightlist
\item
  Core - All modules related to measurement data management
\item
  Extensions - Contains modules for visualisations, hardware sensor
  access
\end{itemize}

In each of these categories there are then several sub-categories
divided into User Interaction Points. An overview of these is given in
the subchapters as the following. There are also introductory examples
which provide an overview of the basic functions in the \emph{Examples}
\ref{examples} section, as well as further examples in the online
documentation \cite{MagneticReadoutProcessingReadTheDocs}.

\hypertarget{core-modules}{%
\subsection{Core Modules}\label{core-modules}}

The included core modules are essential for using the library. Basic
data types are implemented, as well as functions for import and export.
In addition, there are other support scripts that are required
internally.

The following modules are implemented in detail:

\begin{itemize}
\tightlist
\item
  \emph{MRPReading} - storage of measured values
\item
  \emph{MRPMeasurementConfig} - storage of the measurement parameters
\item
  \emph{MRPMagnetTypes} - various physical constants for basic magnet
  types
\end{itemize}

The \emph{MRPReading} module performs a essential role in streamlining
the centralized management of measurement data. It serves as a storage
provider for various measurements, offering functionalities that
facilitate the creation and addition of data records. To customise and
add meta-data, users have the flexibility to configure parameters
through the dedicated \emph{MRPMeasurementConfig} module into an
\emph{MRPReading} instance.

Within the realm of measurement data, a diverse range of data points can
be seamlessly incorporated. The process is initiated by employing
specialized functions designed for the creation and addition of data
records.

To configure parameters, ensuring a tailored approach to the entire
measurement process, these parameters act as a essential bridge between
user preferences and the robust capabilities of the \emph{MRPReading}
module.

The system is also designed to be compatible with \emph{Extension
Modules} \ref{extension-modules}, allowing the generation of measurement
data through various modules. This extensibility enhances the
versatility of the system, accommodating diverse measurement scenarios
and expanding its utility across different domains.

To enhance the accessibility and interpretability of the recorded data,
a dedicated module, \emph{MRPMagnetTypes}, comes into play. This module
is specifically designed for the storage of physical parameters
pertaining to the magnets targeted for measurement.

By centralizing this information, users can streamline the subsequent
phases of evaluation and analysis, simplifying the overall process and
ensuring a more efficient and insightful exploration of the collected
data.

At the end of processing, the collected and modified data are typically
exported; various functions are provided for this purpose. This process
is described in the following subchapter.

\hypertarget{storage-and-datamanagement}{%
\subsection{Storage and
Datamanagement}\label{storage-and-datamanagement}}

An important aspect is data import and export. On the one hand, this
allows the library to reuse and archive the measurements. On the other
hand, the focus during development is that it is also possible to use
the data in other programs.

For this purpose, an open, documented export format must be selected.
Ideally, this should be human-readable and viewable with a simple text
editor. This eliminates all binary-based formats such as the Python
pickle built into Python.

Taking these points into account, the \gls{json} format is chosen. This
is human and machine readable and there is a compatible parser for
almost every programming language.

The following code snippet \ref{lst:json_export_format_example} shows
the \gls{json} structure which is generated when a measurement that
using the library is exported. It can be seen that by using the
\gls{json} format, all measurement points and metadata are available in
readable plain text.

This means that they can also be read out in other programs. Using
serialization, the \emph{MRPReading} class inherited from
Python-\emph{Object} class is serialized via an dictionary conversion
step. This \gls{json} string can then be processed directly or written
to the file system as a file.

\begin{lstlisting}[caption={JSON export structure of an MRPReading based measurement}, label=lst:json_export_format_example]
{
  "time_start": "Wed Sep 20 08:50:13 2023",
  "time_end": "Wed Sep 20 08:54:13 2023",
  "additional_data": {
      "sensor_device_path": "/dev/ttyUnifiedSensorSingle",
      "sensor_name": "Unified Sensor Single Sensor",
      "sensor_id": "386731533439",
      "sensor_capabilities": ["static", "axis_b", "axis_x", "axis_y", "axis_z", "axis_temp", "axis_stimestamp"],
      "configname": "calibrationtemp30.yaml",
      "runner": "cli"
  },
  "data": [{
      "value": 0.135,
      "is_valid": true,
      "id": 0,
      "temperature": 34.32
  }],
  "measurement_config": {
      "id": "525771256544952",
      "sensor_distance_radius": 40.0,
      "magnet_type": 0
  },
  "name": "calibrationtemp30",
}
\end{lstlisting}

The exported example \ref{lst:json_export_format_example} contains the
following different object keys, which contain the following
information:

\begin{itemize}
\tightlist
\item
  \emph{additional\_data} - Additional user-defined metadata
\item
  \emph{data} - Datapoint storage consists of measured value, \gls{uuid}
  and temperature, among other parameters
\item
  \emph{measurement\_config} - Information about the sensor used for
  measurement
\end{itemize}

In addition, further custom objects can be inserted into the \gls{json}
using the functions provided.

Since there are popular data processing frameworks such as \emph{Numpy}
\cite{harris2020array}, or the program for mathematical
calculations, \emph{MATLAB} are often used, the library also supports
export formats for these systems.

The different formats can be triggered by the user by calling up the
corresponding \emph{MRPReading} class functions:

\begin{itemize}
\tightlist
\item
  *.dump()` - \gls{json}
\item
  \emph{.to\_numpy\_matrix()} - \emph{Numpy}-Array of \emph{data} object
  with different options
\item
  \emph{.dump\_savemat()} - \emph{MATLAB} mat-file with measurement
  values and temperatures
\end{itemize}

Currently, data re-import of an exported measurement is only supported
via the \gls{json} format, as an export using the other options
(\emph{Numpy}, \emph{MATLAB})loses data during the export procedure.

\hypertarget{extension-modules}{%
\subsection{Extension Modules}\label{extension-modules}}

The extension modules build on the core modules and offer the user
additional basic functionalities. These include functions for data
acquisition, visualisation and analysis.

\hypertarget{sensor-interface}{%
\subsubsection{Sensor Interface}\label{sensor-interface}}

Another collection of optional modules provided by the library is the
connection of external hardware sensors. All compatible sensors that are
compatible with the firmware developed in the \emph{Unified Sensor}
\ref{unified-sensor} chapter are supported here. The library provides
the following sensor \gls{hal} modules for this purpose:

\begin{itemize}
\tightlist
\item
  \emph{MRPHal} - Firmware protocol implementation
\item
  \emph{MRPHalLocal} - \gls{usb} sensor interface
\item
  \emph{MRPHalRemote} - Remote sensor interface
\end{itemize}

These provide functions to communicate with a connected hardware sensor
and send commands to it. To generate these and convert the received
measurement data into the appropriate format for the core modules, there
is a suitable module for each sensor type:

\begin{itemize}
\tightlist
\item
  \emph{MRPReadingSourceStatic} - for 1D and 2D sensors such as
  \emph{1D: Single Sensor} \ref{d-single-sensor}
\item
  \emph{MRPReadingSourceFullsphere} - for 3D sensors such as \emph{3D:
  Fullsphere} \ref{d-fullsphere}
\end{itemize}

The decision which of these modules to use is made automatically
depending on the connected hardware. For this purpose, a static function
\emph{createReadingSourceInstance} is implemented in the base class
\emph{MRPReadingSource}, which automatically creates the appropriate
instance based on the sensor capabilites.

\hypertarget{visualisation}{%
\subsubsection{Visualisation}\label{visualisation}}

In order to give the user the possibility to display the recorded data
visually, two modules were created, which can graphically prepare
\emph{MRPReading} instances:

\begin{itemize}
\tightlist
\item
  \emph{MRPVisualization} - different table and graph based plots
\item
  \emph{MRPPolarVisualization} - fullsphere map plots
\end{itemize}

On the one hand, it is possible with the \emph{MRPVisualization} module
to display measurement data as a table or plot (e.g.~stream, line,
point). This makes it possible, for example, to visually identify
outliers or trends in the measurement data. These can also be saved as
an image file. The module is compatible with all measurement data.

In contrast to the \emph{MRPPolarVisualization} module, this provides
functions to create 2D map plots a polar coordinate system. This
requires measurement data with additionally set spatial coordinates.
These can be generated automatically with the \emph{3D: Fullsphere}
\ref{d-fullsphere} sensor, or the user must provide the spatial
information from another source.

\hypertarget{analysis}{%
\subsubsection{Analysis}\label{analysis}}

Data analysis offers the user the greatest flexibility to implement
their own modules. For this reason \emph{MRPAnalysis} contains functions
for calculating the following data analyses, which are compatible with
class instances of \emph{MRPReading}:

\begin{itemize}
\tightlist
\item
  \emph{std\_deviation} - Standard deviation
\item
  \emph{mean} - Mean value
\item
  \emph{variance} - Variance
\item
  \emph{CoG} - Centre of gravity
\item
  \emph{binning} - Distribution of a sample by means of a histogram
\item
  \emph{k-nearest} - K-nearest neighbours
\end{itemize}

In addition, the export function \emph{.to\_numpy\_matrix} enables
further processing of the data in the \emph{Numpy}
\cite{harris2020array} framework, in which many other standard
analysis functions are implemented.

\hypertarget{multi-sensor-setup}{%
\section{Multi Sensor Setup}\label{multi-sensor-setup}}

At the current described scenarios, it is only possible to detect and
use sensors that are directly connected to the host \gls{pc}. It has the
disadvantage that there must always be a physical connection. This can
make it difficult to install multiple sensors in measurement setups
where space or cable routing options are limited.

Multiple sensor can be connected to any \gls{pc} which is available on
the network. This can be a \gls{sbc} (e.g.~a Raspberry Pi). The small
footprint and low power consumption make it a good choice. It can also
be used in a temperature chamber.

The \emph{MRPProxy} \ref{MRPlib_Proxy_Module.png} module has been
developed to allow forwarding and interaction with several sensors over
a network connection using a \gls{rest} interface.

The approach of implementing this via a \gls{rest} interface also offers
the advantage that several measurements or experiments can be recorded
at the same time with one remote sensor setup.

\begin{figure}
\centering
\includegraphics{./generated_images/border_MRPlib_Proxy_Module.png}
\caption{MRPlib Proxy Module \label{MRPlib_Proxy_Module.png}}
\end{figure}

Another application example is when sensors are physically separated or
there are long distances between them. By connecting several sensors via
the proxy module, it is possible to link several instances and all
sensors available in the network are available to the \emph{control}
\gls{pc}.

\begin{figure}
\centering
\includegraphics{./generated_images/border_Example_MRP_proxy_module_usage,_using_two_remote_(+pc)s.png}
\caption{Example MRP proxy module usage, using two remote \gls{pc}s
\label{Example_MRP_proxy_module_usage,_using_two_remote_(+pc)s.png}}
\end{figure}

The figure
\ref{Example_MRP_proxy_module_usage,_using_two_remote_(+pc)s.png} shows
the modified \emph{multi-proxy - multi-sensor} topology. Here, both
proxy instances do not communicate directly with the \emph{control}
\gls{pc}, but the proxy instance \emph{remote} \gls{pc} \emph{\#2} can
access the proxy running on \emph{remote} \gls{pc} \emph{\#1}.

The \emph{control} \gls{pc} only communicates with the \emph{remote}
\gls{pc} \emph{\#1}, but can access all sensors in this chain.

\hypertarget{network-proxy}{%
\subsection{Network-Proxy}\label{network-proxy}}

The figure \ref{MRPlib_Proxy_Module.png} shows the separation of the
various \gls{hal} instances, which communicate with the physically
connected sensors on the \emph{remote} \gls{pc} and the \emph{control}
\gls{pc} side, which communicates with the remote side via the network.
For the user, nothing changes in the procedure for setting up a
measurement. The \emph{MRPProxy} \gls{cli} application must always be
started \ref{lst:mrpcli_proxy_start} on the \gls{pc} with connected
hardware sensors attached.

\begin{lstlisting}[language=bash, caption={MRPproxy usage to enable local sensor usage over network}, label=lst:mrpcli_proxy_start]
# START PROXY INSTNACE WITH TWO LOCALLY CONNECTED SENSORS
$ python3 mrpproxy.py proxy launch /dev/ttySENSOR_A /dev/ttySENSOR_B # add another proxy instance http://proxyinstance_2.local for multi-sensor, multi-proxy chain
Proxy started. http://remotepc.local:5556/
PRECHECK: SENSOR_HAL: 1337 # SENSOR A FOUND
PRECHECK: SENSOR_HAL: 4242 # SENSOR B FOUND
Terminate Proxy instance [y/N] [n]: 
\end{lstlisting}

After the proxy instance has been successfully started, it is optionally
possible to check the status via the \gls{rest} interface:
\ref{lst:mrpcli_config_rest}

\begin{lstlisting}[language=bash, caption={MRPProxy REST endpoint query examples}, label=lst:mrpcli_config_rest]
# GET PROXY STATUS
$ wget http://proxyinstance.local:5556/proxy/status
{
"capabilities":[
  "static",
  "axis_b",
  "axis_x",
  "axis_y",
  "axis_z",
  "axis_temp",
  "axis_stimestamp"
],
"commands":[
  "status",
  "initialize",
  "disconnect",
  "combinedsensorcnt",
  "sensorcnt",
  "readsensor",
  "temp"
]}
# RUN A SENSOR COMMAND AND GET THE TOTAL SENSOR COUNT
$ wget http://proxyinstance.local:5556/proxy/command?cmd=combinedsensorcnt
{
"output":[
  "2"
]}
\end{lstlisting}

The query result shows that the sensors are connected correctly and that
their capabilites have also been recognised correctly. To be able to
configure a measurement on the \emph{control} \gls{pc}, only the
\gls{ip} address or hostname of the \gls{pc} running an \emph{MRPProxy}
instance is required \ref{lst:mrpcli_config_using_rpc}.

\begin{lstlisting}[language=bash, caption={MRPcli usage example to connect with a network sensor}, label=lst:mrpcli_config_using_rpc]
# CONFIGURE MEASUREMENT JOB USING A PROXY INSTANCE
$ MRPcli config setupsensor testcfg --path http://proxyinstance.local:5556
> remote sensor connected: True using proxy connection:
> http://proxyinstance.local:5556 with 1 local sensor connected
\end{lstlisting}

\hypertarget{sensor-syncronisation}{%
\subsection{Sensor Syncronisation}\label{sensor-syncronisation}}

Another important aspect when using several sensors via the proxy system
is the synchronisation of the measurement intervals between the sensors.
Individual sensor setups do not require any additional synchronisation
information, as this is communicated via the \gls{usb} interface.

If several sensors are connected locally, they can be connected to each
other via their sync input using short cables. One sensor acts as the
central clock as described in \ref{sensor-syncronisation-interface}.
this no longer works for long distances and the syncronisation must be
made via a shared network connection.

If time-critical synchronisation over the network is required, \gls{ptp}
and \gls{pps} output functionality \cite{PTPIEEE1588} can be used on
many \gls{sbc}, such as the \emph{Raspberry-Pi Compute Module}.

\hypertarget{command-router}{%
\subsection{Command-Router}\label{command-router}}

As it is possible to connect many identical sensors to one host, it must
be possible to address them separately. This separation is done by the
\emph{MRPProxy} module and is a separate part from the core
\gls{mrp}-library, to keep installation package dependencies small.

Each connected sensor is accessed via the text-based \gls{cli}, this is
initially the same for each sensor. The only identification feature is
the sensor \gls{uuid} by using the \emph{id} command of the sensor
\gls{cli}.

The \emph{MRPProxy} instance claims to be a sensor to the host \gls{pc}
running \gls{mrp} \gls{cli}, so the multiple sensors must be combined
into one virtual one. This is done in several steps, start procedure
described by the following sub-chapters.

\hypertarget{construct-the-sensor-id-lookup-table}{%
\subsubsection{Construct the Sensor ID
LookUp-Table}\label{construct-the-sensor-id-lookup-table}}

Immediately after starting the \emph{MRPProxy}, the \gls{uuid}s of all
locally connected sensors are read out. These are stored together with
the class instance of the \emph{MRPHal} module in a \gls{lut}. This
makes it possible to address a sensor directly using its \gls{uuid}.

\hypertarget{merging-the-sensor-capabilities}{%
\subsubsection{Merging the Sensor
Capabilities}\label{merging-the-sensor-capabilities}}

\begin{longtable}[]{@{}llll@{}}
\caption{Sensor capabilities merging algorithm
\label{Sensor_capabilities_merging_algorithm.csv}}\tabularnewline
\toprule
SENSOR A & SENSOR B & MERGED CAPABILITIES & CAPABLE SENSORS ID
LUT\tabularnewline
\midrule
\endfirsthead
\toprule
SENSOR A & SENSOR B & MERGED CAPABILITIES & CAPABLE SENSORS ID
LUT\tabularnewline
\midrule
\endhead
static & & static & A\tabularnewline
& dynamic & dynamic & B\tabularnewline
axis\_temp & axis\_temp & axis\_temp & A B\tabularnewline
axis\_x & axis\_x & axis\_x & A B\tabularnewline
\bottomrule
\end{longtable}

When using sensors with different capabilites, these must be combined.
These are used to select the appropriate measurement mode for a
measurement. For this purpose, the \emph{info} command of each sensor is
queried. This information is added to the previously created \gls{lut}.
Duplicate entries are summarised (see Table
\ref{Sensor_capabilities_merging_algorithm.csv}) and returned to the
host when the \emph{info} \ref{lst:mtsc} command is received over
network.

\begin{lstlisting}[language=bash, caption={MRPproxy REST enpoiint query examples}, label=lst:mtsc]
# QUERY Network-Proxy capabilities
$ wget http://proxyinstance.local:5556/proxy/status
{"capabilities":[
"static",
"dynamic",
"axis_temp",
"axis_x"
]}
\end{lstlisting}

The same procedure is performed for the \emph{commands}
\gls{cli}-command of each sensor, to merge available commands of
connected sensors into the \gls{lut}.

\hypertarget{dynamic-extension-of-the-available-network-proxy-commands}{%
\subsubsection{Dynamic extension of the available Network-Proxy
Commands}\label{dynamic-extension-of-the-available-network-proxy-commands}}

In order for the host to be able to send a command to the network
multi-sensor setup, the command received must be forwarded to the
correct sensor. In addition, there are commands such as the previously
used \emph{info} or \emph{status} command, which must be intercepted by
the \emph{MRPProxy} module so that it can be handled differently (see
example \ref{lst:mtsc}).

To realize this, a \gls{lut} is created in the previous steps, which
contains information regarding \emph{requested capability}
-\textgreater{} \emph{sensor}-\gls{uuid} -\textgreater{} \emph{physical
sensor} and allows the commands to be routed.

For commands where there are several entries for \emph{CAPABLE SENSORS
ID LUT} \ref{Sensor_capabilities_merging_algorithm.csv}, there are two
possible approaches to how the command is processed:

\begin{itemize}
\tightlist
\item
  Redirect to each capable sensor
\item
  Extend commands using an id parameter
\end{itemize}

These two methods have been implemented and are applied automatically.
The decision is based on which hardware sensors are connected. In a
setup where only the same sensor variants are connected, \emph{redirect
to each capable sensor} is applied. This offers a time advantage as
fewer commands need to be sent from the host. Thus, with a
\emph{readsensor} command, all sensors are read out via one command and
the summarized result is transmitted to the host.

The \emph{extend commands using an id parameter} strategy is used for
different sensors. Each command is extended on the \emph{Network-Proxy}
\ref{network-proxy} by another \gls{uuid} parameter, according to the
following scheme:

\begin{itemize}
\tightlist
\item
  \emph{readsensor } -\textgreater{} \emph{readsensor }
\item
  \emph{opmode} -\textgreater{} \emph{opmode }
\end{itemize}

This allows the host to address individual sensors directly via their
specific \gls{uuid}.

\hypertarget{examples}{%
\section{Examples}\label{examples}}

The following shows some examples of how the \gls{mrp}-library can be
used. These examples are limited to a functional minimum for selected
modules of the \gls{mrp}-library. The documentation \ref{documentation}
contains further and more detailed examples. Many basic examples are
also supplied in the form of the test scripts used for testing
\ref{testing}.

\hypertarget{mrpreading}{%
\subsection{MRPReading}\label{mrpreading}}

The \emph{MRPReading} is the key module of the \gls{mrp} core. It is
used to manage the measurement data and can be imported and exported.
The following example \ref{lst:mrpexample_reading} shows how a
measurement is created and measurement points are added in the form of
\emph{MRPReadingEntry} instances.

An important point is the management of the meta data, which further
describes the measurement. This is realised in the example using the
\emph{set\_additional\_data} function.

\begin{lstlisting}[language=Python, caption={MRPReading example for setting up an basic measurement}, label=lst:mrpexample_reading]
from MRP import MRPReading, MRPMeasurementConfig
# [OPTIONAL] CONFIGURE READING USING MEASUREMENT CONFIG INSTANCE
config: MRPMeasurementConfig = MRPMeasurementConfig
config.sensor_distance_radius(40) # 40mm DISTANCE BETWEEN MAGNET AND SENSOR
config.magnet_type(N45_CUBIC_12x12x12) # CHECK MRPMagnetTypes.py FOR IMPLEMENTED TYPES
# CREATE READING
reading: MRPReading = MRPReading(config)
# ADD METADATA
reading.set_name("example reading")
## ADD FURTHER DETAILS
reading.set_additional_data("description", "abc")
reading.set_additional_data("test-number", 1)
# INSERT A DATAPOINT
measurement = MRPReadingEntry.MRPReadingEntry()
measurement.value = random.random()
reading.insert_reading_instance(measurement, False)
# USE MEASURED VALUES IN OTHER FRAMEWORKS / DATAFORMATS
## NUMPY
npmatrix: np.ndarray = reading.to_numpy_matrix()
## CSV
csv: []= reading.to_value_array()
## JSON
js: dict= reading.dump()
# EXPORT READING TO FILE
reading.dump_to_file("exported_reading.mag.json")
# IMPORT READING
imported_reading: MRPReading = MRPReading()
imported_reading.load_from_file("exported_reading.mag.json")
\end{lstlisting}

Finally, the measurement is exported for archiving and further
processing; various export formats are available. Using the
\emph{dump\_to\_file} function, the measurement can be converted into an
open \gls{json} format.

\hypertarget{mrphal}{%
\subsection{MRPHal}\label{mrphal}}

After generating simple measurements with random values in the previous
example \ref{mrpreading}, the next step is to record real sensor data.
For this purpose, the \emph{MRPHal} module is developed, which can
interact with all \emph{Unified Sensor} \ref{unified-sensor}-compatible
sensors. In the following example \ref{lst:mrpexample_hal}, an \emph{1D:
Single Sensor} \ref{d-single-sensor} is connected locally to the host
\gls{pc}.

\begin{lstlisting}[language=Python, caption={MRPHal example to use an connected hardware sensor to store readings inside of a measurement}, label=lst:mrpexample_hal]
from MRP import MRPHalSerialPortInformation, MRPHal, MRPBaseSensor, MRPReadingSource
# SEARCH FOR CONNECTED SENSORS
## LISTS LOCAL CONNECTED OR NETWORK SENSORS
system_ports = MRPHalSerialPortInformation.list_sensors()
sensor = MRPHal(system_ports[0])
# OR USE SPECIFIED SENSOR DEVICE
device_path = MRPHalSerialPortInformation("UNFSensor1")
sensor = MRPHal(device_path)
# RAW SENSOR INTERACTION MODE
sensor.connect()
basesensor = MRPBaseSensor.MRPBaseSensor(sensor)
basesensor.query_readout()
print(basesensor.get_b()) # GET RAW MEASUREMENT
print(basesensor.get_b(1)) # GET RAW DATA FROM SENSOR WITH ID 1
# TO GENERATE A READING THE perform_measurement FUNCTION CAN BE USED
reading_source = MRPReadingSourceHelper.createReadingSourceInstance(sensor)
result_readings: [MRPReading] = reading_source.perform_measurement(_readings=1, _hwavg=1)
\end{lstlisting}

In general, a sensor can be connected using its specific system path or
the sensor-\gls{uuid} via the \emph{MRPHalSerialPortInformation}
function. Locally connected or network sensors can also be automatically
recognised using the \emph{list\_sensors} function. Once connected,
these are then converted into a usable data source using the
\emph{MRPReadingSource} module. This automatically recognises the type
of sensor and generated an \emph{MRPReading} instance with the measured
values of the sensor.

\hypertarget{mrpsimulation}{%
\subsection{MRPSimulation}\label{mrpsimulation}}

If no hardware sensor is available or for the generation of test data,
the \emph{MRPSimulation} module is available. This contains a series of
functions that simulate various magnets and their fields. The result is
a complete \emph{MRPReading} measurement with a wide range of set meta
data.

The example \ref{lst:mrpexample_simulation} illustrated the basic usage.
Different variations of the \emph{generate\_reading} function offers the
user additional parameterisation options, such as random polarisation
direction or a defined centre-of-gravity vector. The data is generated
in the background using the \emph{magpylib}
\cite{ortner2020magpylib} library according to the specified
parameters.

\begin{lstlisting}[language=Python, caption={MRPSimulation example illustrates the usage of several data analysis functions}, label=lst:mrpexample_simulation]
from MRP import MRPSimulation, MRPPolarVisualization, MRPReading
# GENERATE SILUMATED READING USING A SIMULATED HALLSENSOR FROM magpy LIBRARY
reading = MRPSimulation.generate_reading(MagnetType.N45_CUBIC_12x12x12,_add_random_polarisation=True)
# GENERATE A FULLSPHERE MAP READING
reading_fullsphere = MRPSimulation.generate_random_full_sphere_reading()
# RENDER READING TO FILE IN 3D
visu = MRPPolarVisualization(reading)
visu.plot3d(None)
visu.plot3d("simulated_reading.png")
# EXPORT READING
reading.dump_to_file("simulated_reading.mag.json")
\end{lstlisting}

\hypertarget{mrpanalysis}{%
\subsection{MRPAnalysis}\label{mrpanalysis}}

Once data can be acquired using hardware or software sensors, the next
step is to analyse this data. \gls{mrp} provides some simple analysis
functions for this purpose. The code example shows the basic use of the
module. The \emph{Evaluation} \ref{evaluation} chapter shows how the
user can implement their own algorithms and add them to the library.

\begin{lstlisting}[language=Python, caption={MRPAnalysis example code performs several data analysis steps}, label=lst:mrpexample_analysis]
from MRP import MRPAnalysis, MRPReading
# CREATE A SAMPLE MEASUREMENT WITH SIMULATED DATA
reading = MRPSimulation.generate_reading(MagnetType.N45_CUBIC_10x10x10)
# CALCULATE MEAN
print(MRPAnalysis.calculate_mean(reading))
# CALCULATE STD DEVIATION ON TEMPERATURE AXIS
print(MRPAnalysis.calculate_std_deviation(reading, _temperature_axis=True))
# CALCULATE CENTER OF GRAVITY
(x, y, z) = MRPAnalysis.calculate_center_of_gravity(reading)
# APPLY CALIBRATION READING TO REMOVE BACKGROUND NOISE
calibration_reading = MRPSimulation.generate_reading(MagnetType.N45_CUBIC_10x10x10, _ideal = True)
MRPAnalysis.apply_calibration_data_inplace(calibration_reading, reading)
\end{lstlisting}

\hypertarget{mrpvisualisation}{%
\subsection{MRPVisualisation}\label{mrpvisualisation}}

This final example shows the use of the \emph{MRPVisualisation} module,
which provides general functions for visualising measurements. The
visualisation options make it possible to visually assess the results of
a measurement. This is particularly helpful for full-sphere measurements
recorded with the \emph{3D: Fullsphere} \ref{d-fullsphere} sensor. The
sub-module \emph{MRPPolarVisualisation} is specially designed for these.
The figure
\ref{Example_full_sphere_plot_of_an_measurement_using_the_MRPVisualisation_module.png}
shows a plot of a fullsphere measurement. It is also possible to export
the data from the \emph{MRPAnalysis} module graphically as diagrams. The
\emph{MRPVisualisation} modules are used here. The following example
\ref{lst:mrpexample_visualisation} shows the usage of both modules.

\begin{lstlisting}[language=Python, caption={MRPVisualisation example which plots a fullsphere to an image file}, label=lst:mrpexample_visualisation]
from MRP import MRPPolarVisualization
# CREATE MRPPolarVisualization INSTANCE
## IT CAN BE REUSED CALLING plot2d AGAIN, AFTER LINKED READING DATA WERE MODIFIED
visu = MRPPolarVisualization.MRPPolarVisualization(reading)
# 2D PLOT INTO A WINDOW
visu.plot2d_top(None)
visu.plot2d_side(None)
# 3D PLOT TO FILE
visu.plot3d(os.path.join('./plot3d_3d.png'))
# PLOT ANALYSIS RESULTS
from MRP import MRPDataVisualization
MRPDataVisualization.MRPDataVisualization.plot_error([reading_a, reading_b, reading_c])
\end{lstlisting}

\begin{figure}
\centering
\includegraphics{./generated_images/border_Example_full_sphere_plot_of_an_measurement_using_the_MRPVisualisation_module.png}
\caption{Example full sphere plot of an measurement using the
MRPVisualisation module
\label{Example_full_sphere_plot_of_an_measurement_using_the_MRPVisualisation_module.png}}
\end{figure}

\hypertarget{mrphalbacharraygenerator}{%
\subsection{MRPHalbachArrayGenerator}\label{mrphalbacharraygenerator}}

The following example code \ref{lst:mrpexample_halbach}, shows how a
simple Halbach magnetic ring can be generated.

This can then be used to construct a Halbach ring magnet (see chapter
\ref{magnet-system}) for a low-field \gls{mri}.

Eight random measurements are generated here. It is important that the
magnet type (for example \emph{N45\_CUBIC\_15x15x15}) is specified. This
is necessary so that the correct magnet cutouts can be generated when
creating the 3D model.

After the measurements have been generated, they are provided with a
position and rotation offset according to the Halbach design and
calculation scheme \cite{HallbachMagnetDesignPaper} using the
\emph{MRPHalbachArrayGenerator} module.

\begin{lstlisting}[language=Python, caption={MRPHalbachArrayGenerator example for generating an OpenSCAD based halbach ring}, label=lst:mrpexample_halbach]
readings = []
for idx in range(8):
  # GENERATE EXAMPLE READINGS USING N45 CUBIC 15x15x15 MAGNETS
  readings.append(MRPSimulation.MRPSimulation.generate_reading(MRPMagnetTypes.MagnetType.N45_CUBIC_15x15x15))
## GENERATE HALBACH
halbach_array: MRPHalbachArrayGenerator.MRPHalbachArrayResult = MRPHalbachArrayGenerator.MRPHalbachArrayGenerator.generate_1k_halbach_using_polarisation_direction(readings)
# EXPORT TO OPENSCAD
## 2D MODE DXF e.g. for lasercutting
MRPHalbachArrayGenerator.MRPHalbachArrayGenerator.generate_openscad_model([halbach_array], "./2d_test.scad",_2d_object_code=True)
## 3D MODE e.g. for 3D printing
MRPHalbachArrayGenerator.MRPHalbachArrayGenerator.generate_openscad_model([halbach_array], "./3d_test.scad",_2d_object_code=False)
\end{lstlisting}

In the last step, a 3D model with the dimensions of the magnet type set
is generated from the generated magnet positions. The result is an
\emph{OpenSCAD} \cite{OpenSCAD} file, which contains the module
generated. After computing the model using the \emph{OpenSCAD} \gls{cli}
utility, the following model rendering
\ref{Generated_Hallbach_array_with_generated_cutouts_for_eight_magnets.png}
can be generated.

\begin{figure}
\centering
\includegraphics{./generated_images/border_Generated_Hallbach_array_with_generated_cutouts_for_eight_magnets.png}
\caption{Generated Hallbach array with generated cutouts for eight
magnets
\label{Generated_Hallbach_array_with_generated_cutouts_for_eight_magnets.png}}
\end{figure}

\hypertarget{usability-improvements}{%
\chapter{Usability Improvements}\label{usability-improvements}}

Usability improvements in software libraries are essential for efficient
and user-friendly development. Intuitive API documentation, clearly
structured code examples and improved error messages promote a smooth
developer experience. A \gls{gui} or \gls{cli} application for complex
libraries can make it easier to use, especially for developers with less
experience. Continuous feedback through automated tests and
comprehensive error logs enable faster bug fixing.

The integration of user feedback and regular updates promotes the
adaptability of the \gls{mrp}-library. Effective usability improvements
help to speed up development processes and increase the satisfaction of
the developer community. In the following, some of these have been added
in and around the \gls{mrp}-library, but they are only optional
components for the intended use.

\hypertarget{command-line-interface}{%
\section{Command Line Interface}\label{command-line-interface}}

\begin{figure}
\centering
\includegraphics{./generated_images/border_MRP_(+cli)_output_to_configure_a_new_measurement.png}
\caption{MRP \gls{cli} output to configure a new measurement
\label{MRP_(+cli)_output_to_configure_a_new_measurement.png}}
\end{figure}

In the first version of the \gls{mrp}-library, the user had to write his
own Python scripts even for short measurement and visualisation tasks.
This is already a time-consuming process for reading out a sensor and
configuring the measurement parameters and metadata and quickly required
more than 100 lines of new Python code.

Although such examples are provided in the documentation, it must be
possible for programming beginners in particular to use them. To
simplify these tasks, a \gls{cli}
\ref{Example_measurement_analysis_pipeline.png} is implemented. The
library \gls{cli} implements the following functionalities:

\begin{itemize}
\tightlist
\item
  Detection of connected sensors
\item
  Configuration of measurement series
\item
  Recording of measured values from stored measurement series
\item
  Simple commands for checking recorded measurement series and their
  data
\end{itemize}

Thanks to this functionality of the \gls{cli}, it is now possible to
connect a sensor to the \gls{pc}, configure a measurement series with it
and run it at the end. The result is an exported file with the measured
values. These can then be read in again using the \emph{MRPReading}
module and processed further. The following bash code
\ref{lst:mrpcli_config_run} shows the setup procedure in detail:

\begin{lstlisting}[language=bash, caption={CLI example for configuring a measurement run}, label=lst:mrpcli_config_run]
# CLI EXAMPLE FOR CONFIGURING A MEASUREMENT RUN
## CONFIGURE THE SENSOR TO USE
$ MRPcli config setupsensor testcfg
> 0 - Unified Sensor 386731533439 - /dev/cu.usbmodem3867315334391
> Please select one of the found sensors [0]:
> sensor connected: True 1243455
## CONFIGURE THE MEASUREMENT
$ MRPcli config setup testcfg
> CONFIGURE testcfg
> READING-NAME: [testreading]: testreading
> OUTPUT-FOLDER [/cli/reading]: /tmp/reading_folder_path
> NUMBER DATAPOINTS: [1]: 10
> NUMBER AVERAGE READINGS PER DATAPOINT: [1]: 100
# RUN THE CONFIGURED MEASUREMENT
$ MRPcli measure run
> STARTING MEASUREMENT RUN WITH FOLLOWING CONFIGS: ['testcfg']
> config-test: OK
> sensor-connection-test: OK
> START MEASUREMENT CYCLE
> sampling 10 datapoints with 100 average readings
> SID:0 DP:0 B:47.359mT TEMP:23.56
> ....
> dump_to_file testreading_ID:525771256544952_SID:0_MAG:N45_CUBIC_12x12x12.mag.json
\end{lstlisting}

\hypertarget{programmable-data-processing-pipeline}{%
\section{Programmable Data Processing
Pipeline}\label{programmable-data-processing-pipeline}}

After it is easy for users to carry out measurements using the
\gls{cli}, the next logical step is to analyse the recorded data. This
can involve one or several hundred data records. Again, the procedure
for the user is to write their own evaluation scripts using the
\gls{mrp}-library.

This is particularly useful for complex analyses or custom algorithms,
but not necessarily for simple standard tasks such as bias compensation
or graphical plot outputs.

\begin{figure}
\centering
\includegraphics{./generated_images/border_Example_measurement_analysis_pipeline.png}
\caption{Example measurement analysis pipeline
\label{Example_measurement_analysis_pipeline.png}}
\end{figure}

For this purpose, a further \gls{cli} application is created, which
enables the user to create and execute complex evaluation pipelines for
measurement data without programming. The example
\ref{Example_measurement_analysis_pipeline.png} shows a typical
measurement data analysis pipeline, which consists of the following
steps:

\begin{enumerate}
\def\labelenumi{\arabic{enumi}.}
\tightlist
\item
  Import the measurements
\item
  Determine sensor bias value from imported measurements using a
  reference measurement
\item
  Apply linear temperature compensation
\item
  Export the modified measurements
\item
  Create a graphical plot of all measurements with standard deviation
\end{enumerate}

In order to implement such a pipeline, the \emph{yaml} file format is
chosen for the definition of the pipeline, as this is for non
programmers to understand and can also be easily edited with a plain
text editor. Detailed examples can be found in the documentation
\cite{MagneticReadoutProcessingReadTheDocs}. The pipeline definition
consists of sections which execute the appropriate Python commands in
the background.

The signatures in the \emph{yaml} file are called using reflection and a
real-time search of the loaded \emph{global()} functions symbol table
\cite{PythonGlobalSymbolTable}. This system makes almost all Python
functions available to the user. To simplify use, a pre-defined list of
verified functions for use in pipelines is listed in the documentation
\cite{MagneticReadoutProcessingReadTheDocs}. The following pipeline
definition \ref{lst:mrpuddp_example_yaml} shows the previously defined
steps \ref{Example_measurement_analysis_pipeline.png} as \emph{yaml}
syntax.

\begin{lstlisting}[caption={Example YAML code of an user defined processing pipeline with six stages linked together}, label=lst:mrpuddp_example_yaml]
stage import_readings:
  function: import_readings
  parameters:
    IP_input_folder: ./readings/fullsphere/
    IP_file_regex: 360_(.)*.mag.json

stage import_bias_reading:
  function: import_readings
  parameters:
    IP_input_folder: ./readings/fullsphere/
    IP_file_regex: bias_reading.mag.json

stage apply_bias_offset:
  function: apply_sensor_bias_offset
  parameters:
    bias_readings: stage import_bias_reading # USE RESULT FROM FUNCTION import_bias_reading
    readings_to_calibrate: stage import_readings

stage apply_temp_compensation:
  function: apply_temperature_compensation
  parameters:
    readings_to_calibrate: stage import_readings # USE RESULT FROM FUNCTION import_readings

stage plot_normal_bias_offset:
  function: plot_readings
  parameters:
    readings_to_plot: stage apply_temp_compensation
    IP_export_folder: ./readings/fullsphere/plots/
    IP_plot_headline_prefix: Sample N45 12x12x12 magnets calibrated

stage export_readings:
  function: export_readings
  parameters:
    readings_to_plot: stage apply_temp_compensation
    IP_export_folder: ./readings/fullsphere/plots/
\end{lstlisting}

Each pipeline step is divided into \emph{stages}, which contain a name,
the function to be executed and its parameters.

\newpage

The various steps are then linked by using the \emph{stage } makro as
input parameter of the next function to be executed (see comments in
\ref{lst:mrpuddp_example_yaml}).

It is therefore also possible to use the results of one function in
several others without them directly following each other. The
disadvantages of this system are the following:

\begin{itemize}
\tightlist
\item
  No circular parameter dependencies
\item
  Complex determination of the execution sequence of the steps
\end{itemize}

To determine the order of the pipeline steps, the parser script created
converts them into one problem of the graph theories. Each step
represents a node in the graph and the steps referred to by the input
parameter form the edges. After several simplification steps,
determination of possible start steps and repeated traversal, the final
execution sequence can be determined in the form of a call tree
\ref{Example_result_of_an_step_execution_tree_from_user_defined_processing_pipeline.png}.
The individual steps are then executed along the graph. The intermediate
results and the final results
\ref{Pipeline_output_files_after_running_example_pipeline_on_a_set_of_readings.png}
are saved for optional later use.

\begin{figure}
\centering
\includegraphics{./generated_images/border_Example_result_of_an_step_execution_tree_from_user_defined_processing_pipeline.png}
\caption{Example result of an step execution tree from user defined
processing pipeline
\label{Example_result_of_an_step_execution_tree_from_user_defined_processing_pipeline.png}}
\end{figure}

\begin{figure}
\centering
\includegraphics{./generated_images/border_Pipeline_output_files_after_running_example_pipeline_on_a_set_of_readings.png}
\caption{Pipeline output files after running example pipeline on a set
of readings
\label{Pipeline_output_files_after_running_example_pipeline_on_a_set_of_readings.png}}
\end{figure}

\hypertarget{testing}{%
\section{Testing}\label{testing}}

\begin{figure}
\centering
\includegraphics{./generated_images/border_MRP_library_test_results_for_different_submodules_executed_in_PyCharm_(+ide).png}
\caption{MRP library test results for different submodules executed in
PyCharm \gls{ide}
\label{MRP_library_test_results_for_different_submodules_executed_in_PyCharm_(+ide).png}}
\end{figure}

Software tests in libraries offer numerous advantages for improving
quality and efficiency. Software tests make it possible to identify
errors and vulnerabilities before the software is released as a new
version.

This significantly improves the reliability of \gls{mrp}-library
applications. Tests also ensures consistent and reliable performance,
which is particularly important when libraries are used by different
users and for different usecases.

During the development of the \gls{mrp}-library, test cases were also
created for all important functionalities and usecases. The test
framework \emph{PyTest} \cite{PyTest} is used for this purpose, as
it offers direct integration in most \gls{ide}s (see
\ref{MRP_library_test_results_for_different_submodules_executed_in_PyCharm_(+ide).png})
and also because it provides detailed and easy-to-understand test
reports as output in order to quickly identify and correct errors. It
also allows to tag tests, which is useful for grouping tests or
excluding certain tests in certain build environment scenarios. Since
all intended usecases were mapped using the test cases created, the code
of the test cases could later be used in slightly simplified variants
\ref{lst:pytest_example_code} as examples for the documentation.

\begin{lstlisting}[language=Python, caption={Example pytest class for testing MRPReading module functions}, label=lst:pytest_example_code]
class TestMPRReading(unittest.TestCase):
  # PREPARE A INITIAL CONFIGURATION FILE FOR ALL FOLLOWING TEST CASES IN THIS FILE
  def setUp(self) -> None:
    self.test_folder: str = os.path.join(os.path.dirname(os.path.abspath(__file__)), "tmp")
    self.test_file:str = os.path.join(self.import_export_test_folderpath, "tmp")

  def test_matrix(self):
    reading: MRPReading = MRPSimulation.generate_reading()
    matrix: np.ndarray = reading.to_numpy_matrix()
    n_phi: float = reading.measurement_config.n_phi
    n_theta: float = reading.measurement_config.n_theta
    # CHECK MATRIX SHAPE
    self.assertTrue(matrix.shape != (n_theta,))
    self.assertTrue(len(matrix.shape) <= n_phi)

  def test_export_reading(self) -> None:
    reading: MRPReading = MRPSimulation.generate_reading()
    self.assertIsNotNone(reading)
    # EXPORT READING TO A FILE
    reading.dump_to_file(self.test_file)

  def test_import_reading(self):
    # CREATE EMPTY READING AND LOAD FROM FILE
    reading_imported:MRPReading = MRPReading.MRPReading(None)
    reading_imported.load_from_file(self.test_file)
    # COMPARE
    self.assertIsNotNone(reading_imported.compare(MRPSimulation.generate_reading()))
\end{lstlisting}

One problem, is the parts of the \gls{mrp}-library that require direct
access to external hardware. These are, for example, the \emph{MRPHal}
and \emph{MRPHalRest} modules, which are required to read out sensors
connected via the network. Two different approaches were used here. In
the case of local development, the test runs were carried out on a
\gls{pc} that can reach the network hardware and thus the test run could
be carried out with real data.

\newpage

In the other scenario, the tests are to be carried out before a new
release in the repository on the basis of \emph{Github Actions}
\cite{GithubActions}. Here there is the possibility to host local
runner software, which then has access to the hardware, but then a
\gls{pc} must be permanently available for this task. Instead, the
hardware sensors were simulated by software and executed via
virtualisation on the systems provided by \emph{Github Actions}.

\hypertarget{package-distribution}{%
\section{Package Distribution}\label{package-distribution}}

\begin{figure}
\centering
\includegraphics{./generated_images/border_MagneticReadoutProcessing_library_hosted_on_PyPi.png}
\caption{MagneticReadoutProcessing library hosted on PyPi
\label{MagneticReadoutProcessing_library_hosted_on_PyPi.png}}
\end{figure}

One important point that improves usability for users is the simple
installation of all \gls{mrp} modules. As it is created in the Python
programming language, there are several public package registry where
users can provide their software modules. Here, \emph{PyPi}
\cite{PyPI}
\ref{MagneticReadoutProcessing_library_hosted_on_PyPi.png}
\cite{MagneticReadoutProcessingPyPI} is the most commonly used
package registry and offers direct support for the package installation
program \gls{pip} \ref{lst:setup_lib_with_pip}.

\gls{pip} not only manages possible package dependencies, but also
manages the installation of different versions of a package. In
addition, the version compatibility is also checked during the
installation of a new package, which can be resolved manually by the
user in the event of conflicts.

\begin{lstlisting}[language=bash, caption={Bash commands to install the MagneticReadoutProcessing library using pip}, label=lst:setup_lib_with_pip]
# https://pypi.org/project/MagneticReadoutProcessing/
# install the latest version
$ pip3 install MagneticReadoutProcessing
# install the specific version 1.4.0
$ pip3 install MagneticReadoutProcessing==1.4.0
\end{lstlisting}

To make the \gls{mrp} file structure compatible with the package
registry, Python provides separate installation routines that build a
package in an isolated environment and then provide an installation
\emph{wheel} archive. This can then be uploaded to the package registry.
Since the \gls{mrp}-library requires additional Python dependencies,
which cannot be assumed to be already installed on the target system,
these must be installed prior to the actual installation. These can be
specified in the library installation configuration \emph{setup.py}
\ref{lst:setup_py_req} for this purpose.

\begin{lstlisting}[language=Python, caption={setup.py with dynamic requirement parsing used given requirements.txt}, label=lst:setup_py_req]
# dynamic requirement loading using 'requirements.txt'
req_path = './requirements.txt'
with pathlib.Path(req_path).open() as requirements_txt:
  install_requires = [str(requirement) for requirement in pkg_resources.parse_requirements(requirements_txt)]

setup(name='MagneticReadoutProcessing',
  version='1.4.3',
  url='https://github.com/LFB-MRI/MagnetCharacterization/',
  packages= ['MRP', 'MRPcli', 'MRPudpp', 'MRPproxy'],
  install_requires=install_requires,
  entry_points={
    'console_scripts': [
      'MRPCli = MRPcli.cli:run',
      'MRPUdpp = MRPudpp.uddp:run',
      'MRPproxy = MRPproxy.mrpproxy:run'
    ]
  }
)
\end{lstlisting}

To make the \gls{cli} scripts written in Python easier for the user to
execute without having to use the \emph{python3} prefix. This has been
configured in the installation configuration using the
\emph{entry\_points} option, and the following commands are available to
the user:

\begin{itemize}
\tightlist
\item
  \emph{\$ MRPcli --help} instead of \emph{\$ python3 cli.py --help}
\item
  \emph{\$ MRPudpp --help} instead of \emph{\$ python3 udpp.py --help}
\item
  \emph{\$ MRPproxy --help} instead of \emph{\$ python3 proxy.py --help}
\end{itemize}

In addition, these commands are available globally in the system without
the terminal shell being located in the \gls{mrp}-library folder.

\hypertarget{documentation}{%
\subsection{Documentation}\label{documentation}}

In order to provide comprehensive documentation for the enduser, the
source code is documented using Python-\emph{docstrings}
\cite{PythonDocstringReference} and the Python type annotations. The
use of type annotations also simplifies further development, as modern
\gls{ide}s can more reliably display possible methods to the user as an
assistance. \ref{lst:pydocstring}

\begin{lstlisting}[language=Python, caption={Documentation using Python docstring example}, label=lst:pydocstring]
# MRPDataVisualisation.py - example docstring
def plot_temperature(_readings: [MRPReading.MRPReading], _title: str = '', _filename: str = None, _unit: str = "degree C") -> str:
  """
  Plots a temperature bar graph of the reading data entries as figure
  :param _readings: readings to plot
  :type _readings: list(MRPReading.MRPReading)
  :param _title: Title text of the figure, embedded into the head
  :type _title: str
  :param _filename: export graphic to an given absolute filepath with .png
  :type _filename: str
  :returns: returns the abs filepath of the generated file
  :rtype: str
  """
  if _readings is None or len(_readings) <= 0:
      raise MRPDataVisualizationException("no readings in _reading given")
  num_readings = len(_readings)
  # ...
\end{lstlisting}

Since \emph{docstrings} only document the source code, but do not
provide simple how-to-use instructions, the documentation framework
\emph{Sphinx} \cite{SphinxDocumentation} is used for this purpose.
This framework makes it possible to generate \gls{html} or \gls{pdf}
documentation from various source code documentation formats, such as
the used \emph{docstrings}.

These are converted into a Markdown format in an intermediate step and
this also allows to add further user documentation such as examples or
installation instructions. In order to make the documentation created by
\emph{Sphinx} accessible to the user, there are, as with the package
management by \emph{PyPi} services, which provide the \gls{mrp}-library
documentation online.

Once the finished documentation has been generated from static
\gls{html} files, it is stored in the project repository. Another
publication option is to host the documentation via online services such
as \emph{ReadTheDocs} \cite{ReadTheDocs}, where users can make
documentation for typical software projects available to others.

The documentation has also been uploaded to \emph{ReadTheDocs}
\cite{MagneticReadoutProcessingReadTheDocs} and linked in the
repository and on the overview page
\ref{MagneticReadoutProcessing_documentation_hosted_on_ReadTheDocs.png}
on \emph{PyPi}.

The process of creating and publishing the documentation has been
automated using \emph{GitHub Actions}, so that it is always
automatically kept up to date with new features.

\begin{figure}
\centering
\includegraphics{./generated_images/border_MagneticReadoutProcessing_documentation_hosted_on_ReadTheDocs.png}
\caption{MagneticReadoutProcessing documentation hosted on ReadTheDocs
\label{MagneticReadoutProcessing_documentation_hosted_on_ReadTheDocs.png}}
\end{figure}

\hypertarget{usecase-evaluation}{%
\chapter{Usecase Evaluation}\label{usecase-evaluation}}

The practical application of the hardware and software framework is
shown below. This is shown using the previously defined usecases
\ref{usecases}. In the application example, various permanent magnets
are measured and then sorted according to their field strength. The
result should then list the magnets that deviate the least from each
other in terms of their field strength. This is determined using a
sorting algorithm developed by the user.

The process is broken down into the following steps and the practical
application of \emph{User Interaction Points}
\ref{user-interaction-points} is shown:

\begin{enumerate}
\def\labelenumi{\arabic{enumi}.}
\item
  \textbf{Hardware preparation}: Users can prepare measurements using
  the implemented framework. This includes the placement of the sensors
  and the selection of the relevant parameters for the characterisation
  of the permanent magnets.
\item
  \textbf{Configuration of the measurement}: The software provides a
  user-friendly interface for configuring the measurement parameters.
  Users can make the desired settings here and customise the framework
  to their specific requirements.
\item
  \textbf{Custom algorithm implementation}: An important contribution of
  the \gls{mrp} ecosystem is the possibility for users to implement
  their own algorithm for data analysis. This allows customisation to
  specific research questions or experimental requirements.
\item
  \textbf{Execution of analysis pipeline}: The analysis pipeline can
  then be executed with the implemented algorithm. The collected
  measurement data is automatically processed and analysed to extract
  characteristic parameters of the permanent magnets.
\end{enumerate}

This process covers all the essential functionalities required for a
comprehensive characterisation of permanent magnets. These were
previously described in the Usecases \ref{usecases} chapter. The
developed framework not only offers a cost-effective and flexible
hardware solution, but also enables customisation of the analysis
algorithms to meet the requirements of different research projects.

\hypertarget{hardware-preparation}{%
\section{Hardware preparation}\label{hardware-preparation}}

\begin{figure}
\centering
\includegraphics{./generated_images/border_Ten_numbered_test_magnets_in_separate_holders.png}
\caption{Ten numbered test magnets in separate holders
\label{Ten_numbered_test_magnets_in_separate_holders.png}}
\end{figure}

For the hardware setup, the 3D-Fullsphere\ref{d-fullsphere} sensor is
used for the evaluation of the framework. As this is equipped with an
exchangeable magnetic holder mount, suitable holders are required for
the magnets to be measured. Ten random \emph{N45 12x12x12mm} neodymium
magnets were used, which are shown in figure
\ref{Ten_numbered_test_magnets_in_separate_holders.png}.

These were placed in modified 3D printed holders
\ref{Ten_numbered_test_magnets_in_separate_holders.png} and then
numbered. This allows them to be matched to the measurement results
later.

\hypertarget{configuration-of-the-measurement}{%
\section{Configuration of the
Measurement}\label{configuration-of-the-measurement}}

The configured hardware is then connected to the host system using the
\emph{MRPcli config setupsensor}-\gls{cli} command. Afterwards, the
measurement is configured for an measurement run, using the following
configuration commands \ref{lst:evaluation_measurement_config}.

\newpage

\begin{lstlisting}[language=bash, caption={Measurement configuration for evaluation measurement}, label=lst:evaluation_measurement_config]
## CONFIGURE THE MEASUREMENT
$ MRPcli config setup eval_measurement_config
> READING-NAME: 360_eval_magnet_<id>
> OUTPUT-FOLDER: ./readings/evaluation/ 
> NUMBER DATAPOINTS: 18 # FOR A FULLSPHERE READING USE MULTIBLE OF 18
> NUMBER AVERAGE READINGS PER DATAPOINT: 10
\end{lstlisting}

The \emph{MRPcli measure run} command is then called up for each
individual magnet to execute a measurement. After each run, the
\emph{READING-NAME} parameter is filled with the id of the next magnet
so that all measurements could be assigned to the physical magnets.

\hypertarget{custom-algorithm-implementation}{%
\section{Custom Algorithm
Implementation}\label{custom-algorithm-implementation}}

The next step for the user is the implementation of the filter algorithm
\ref{lst:custom_find_similar_values_algorithm}. This can have any
function signature and is implemented in the file
\emph{UDPPFunctionCollection.py}. This Python file is loaded when the
pipeline is started and all functions that are imported here as a module
or implemented directly can be called via the pipeline. As this is a
short algorithm, it is inserted directly into the file.

The parameter *\_readings* should later receive the imported
measurements from the \emph{stage rawimport}
\ref{lst:pipeline_mrp_evaluation_yaml} and the optional
\emph{IP\_return\_count} parameter specifies the number of best
measurements that are returned. The return parameter is a list of
measurements containing the most similar measurements, measured by the
smallest distance between all measurements. The distance for each
measurement is determined using the centre of gravity function
\emph{CoG}, the length is then calculated from the result vector. This
value can then be used for sorting.

\begin{lstlisting}[language=Python, caption={User implemented custom find most similar readings algorithm}, label=lst:custom_find_similar_values_algorithm]
@staticmethod
def FindSimilarValuesAlgorithm(_readings: [MRPReading.MRPReading], IP_return_count: int = -1) -> [MRPReading.MRPReading]:
  import heapq
  import math
  heap = []
  # SET RESULT VALUE COUNT
  IP_return_count = max([int(IP_return_count),len(_readings)])
  if IP_return_count < 0:
      IP_return_count = len(_readings) / 5
  # CALCULATE TARGET VALUE: CENTER OF GRAVITY MAGNITUDE FOR ALL READINGS
  target_value = 0.0
  for idx, r in enumerate(_readings):
      cog = MRPAnalysis.MRPAnalysis.calculate_center_of_gravity(r)
      # COMPUTE LENGTH OF VECTOR
      cog_value: float = math.sqrt(cog[0]**2 + cog[1]**2 cog[2]**2)
      target_value = target_value + cog
  target_value = target_value / len(_readings)
  # PUSH READINGS TO HEAP
  for value in _readings:
      # USE DIFF AS PRIORITY VALUE IN MIN-HEAP
      cog = MRPAnalysis.MRPAnalysis.calculate_center_of_gravity(value)
      cog_value: float = math.sqrt(cog[0]**2 + cog[1]**2 cog[2]**2)
      diff = abs(cog_value - target_value)
      heapq.heappush(heap, (diff, cog_value))
  # RETURN X BEST ITEMS FROM HEAP
  similar_values = [item[1] for item in heapq.nsmallest(IP_return_count, heap)]
  # CLEAN UP USED LIBRARIES AND RETURN RESULT
  del heapq
  del math
  return similar_values
\end{lstlisting}

The Python \emph{heapq} \cite{heapq} module, which implements a
priority queue, is used for this purpose. The calculated distances from
the \emph{CoG} value of the measurements to are inserted into this
queue. Subsequently, as many elements of the queue are returned as
defined by the \emph{IP\_return\_count} parameter. The actual sorting is
carried out by the queue in the background.

\hypertarget{alternative-filter-algorithm-implementation}{%
\subsection{Alternative Filter Algorithm
Implementation}\label{alternative-filter-algorithm-implementation}}

Another possibility here would be for the user to use a reference
measurement instead of a simulated ideal magnet as a reference. This can
come from a magnet selected as a reference magnet. As a result, the
filter algorithm returns the measurements that are most similar to the
selected reference magnet.The code snipped
\ref{lst:custom_find_similar_values_algorithm_refmagnet} shows the
modified filter algorithm code, with added *\_ref* input parameter for
the reference measurement.

\begin{lstlisting}[language=Python, caption={Modified user implemented custom find algorithm using a reference magnet reading}, label=lst:custom_find_similar_values_algorithm_refmagnet]
@staticmethod
def FindSimilarValuesAlgorithmREF(_readings: [MRPReading.MRPReading], _ref: [MRPReading.MRPReading], IP_return_count: int = 4) -> [MRPReading.MRPReading]:
  # CALCULATE CoG VALUE OF REFERENCE MAGNET
  cog_ref = MRPAnalysis.MRPAnalysis.calculate_center_of_gravity(_ref[0])
  ref_value: float = math.sqrt(cog_ref[0]**2 + cog_ref[1]**2 cog_ref[2]**2)
  # PUSH READINGS TO HEAP
  for value in _readings:
      # USE DIFF AS PRIORITY VALUE IN MIN-HEAP
      cog = MRPAnalysis.MRPAnalysis.calculate_center_of_gravity(value)
      cog_value: float = math.sqrt(cog[0]**2 + cog[1]**2 cog[2]**2)
      heapq.heappush(heap, (abs(cog_value - ref_value), cog_value))
  # RETURN X BEST ITEMS FROM HEAP
  return [item[1] for item in heapq.nsmallest(IP_return_count, heap)]
\end{lstlisting}

\hypertarget{execution-of-analysis-pipeline}{%
\section{Execution of Analysis
Pipeline}\label{execution-of-analysis-pipeline}}

Once the filter function has been implemented, it still needs to be
integrated into the analysis
pipeline\ref{lst:pipeline_mrp_evaluation_yaml}. Here, the example
pipeline \ref{Example_measurement_analysis_pipeline.png} is simplified
and an additional stage \emph{find\_similar\_values} has been added,
which has set \emph{FindSimilarValuesAlgorithm} as the function to be
called. As a final step, the result is used in the \emph{plot\_filtered}
stage for visualisation.

\begin{lstlisting}[caption={User defined processing pipeline using custom implemented filter algorithm}, label=lst:pipeline_mrp_evaluation_yaml]
settings:
  enabled: true
  export_intermediate_results: false
  name: pipeline_mrp_evaluation

stage rawimport:
  function: import_readings
  parameters:
    IP_input_folder: ./readings/evaluation/
    IP_file_regex: 360_(.)*.mag.json

stage find_similar_values:
  function: custom_find_similar_values_algorithm
  parameters:
    _readings: stage rawimport # USE RESULTS FROM rawimport STAGE
    IP_return_count: 4 # RETURN BEST 4 of 10 READINGS

stage plot_filtered:
  function: plot_readings
  parameters:
    readings_to_plot: stage find_similar_values # USE RESULTS FROM find_similar_values STAGE
    IP_export_folder: ./readings/evaluation/plots/plot_filtered/
    IP_plot_headline_prefix:  MRP evaluation - filtered
\end{lstlisting}

The final pipeline has been saved in the pipeline directory as
\emph{pipeline\_mrp\_evaluation.yaml} file and is ready for execution.
This is carried out using the \emph{MRPudpp} \gls{cli}
\ref{lst:bash_pipeline_mrp_evaluation_yaml}. After the run has been
successfully completed, the results are saved in the result folder
specified in the pipeline using the \emph{IP\_export\_folder} parameter.

\begin{lstlisting}[language=bash, caption={Bash result log of evaluation pipeline run}, label=lst:bash_pipeline_mrp_evaluation_yaml]
# LIST ACTIVE PIPELINES IN PIPELINE DIRECTORY 
$ MRPudpp pipeline listenabledpipelines
> Found enabled pipelines:
> 1. pipeline_mrp_evaluation.yaml
# EXECUTE THE EVALUATION PIPELINE
$ MRPudpp pipeline run
> loading pipeline pipeline_mrp_evaluation.yaml
> stage nodes: ['import', 'find_similar_values', 'plot_raw', 'plot_filtered']
> =====> stage: import 
> =====> stage: find_similar_values 
> =====> stage plot_filtered 
> Process finished with exit code 0
\end{lstlisting}

\hypertarget{result-analysis}{%
\section{Result Analysis}\label{result-analysis}}

The figure
\ref{MRP_evaluation_result_after_execution_of_the _user_defined_pipeline,_using_find_similar_values_algorithm.png}
shows this result. The plot of the raw measured values is shown on the
left. The value of the determined \emph{GoG} \(\mu\)T values is plotted
on ten individual measured values. Here it can be seen that there are
measured values with larger deviations (see measurement
\emph{7:0},\emph{10-2:0},\emph{10-1:0}).

\begin{figure}
\centering
\includegraphics{./generated_images/border_MRP_evaluation_result_after_execution_of_the _user_defined_pipeline,_using_find_similar_values_algorithm.png}
\caption{MRP evaluation result after execution of the user defined
pipeline, using find similar values algorithm
\label{MRP_evaluation_result_after_execution_of_the _user_defined_pipeline,_using_find_similar_values_algorithm.png}}
\end{figure}

On the right-hand side
\ref{MRP_evaluation_result_after_execution_of_the _user_defined_pipeline,_using_find_similar_values_algorithm.png},
the measured values are plotted as a result of the filter algorithm. As
the \emph{IP\_return\_count} parameter is set to four, only the four
most similar measurements were exported here. It can be seen from the
plotted \emph{CoG} \(\mu\)T deviation values, that these are closest to
an ideal Magnet with a CoG value of 0\(\mu\)T. This ideal value is
calculated with the function
\emph{MRP.MRPSimulation.generate\_simulated\_reading}, with the same
measurement parameters (magnet type, dimensions, sensor distance) as
they correspond to the mechanical structure of the used hardware sensor
\ref{d-fullsphere}.

If the alternative filter algorithm from chapter \emph{Alternative
Filter Algorithm Implementation}
\ref{alternative-filter-algorithm-implementation} is executed here, the
same result is returned if the magnet measurement with \gls{id}
\emph{5:0} is used as the reference magnet.

The filter algorithm implemented by the user is thus successfully
executed using the user-programmable pipeline. The calculation result is
successfully verified using raw measurement data and the final result of
the algorithm.

\hypertarget{evaluation}{%
\chapter{Evaluation}\label{evaluation}}

In the previous chapter \emph{Usecase Evaluation}
\ref{usecase-evaluation} it is shown that the implementation of the
hardware and software framework for various magnetic field sensors is
successfully implemented.

In addition, the basic application by the user is demonstrated based on
an example. In this way is possible to systematically characterise
magnets from the software and readout hardware side by means of data
acquisition, storage and analysis.

After discussing the developed hardware and software components in
particular, this chapter will answer the question of whether the
selected sensors meet the requirements:

\begin{itemize}
\tightlist
\item
  Measure a wide range of different permanent magnets with regard to
  their systematic field strength
\item
  Automated measurement of the homogeneity of Halbach rings with an
  accuracy of less than \emph{1000\gls{ppm}}
\end{itemize}

These questions are answered in the following, as a basic readout and
analysis functionality platform had to be created first and thus an
automated sensor characterisation can be performed.

\hypertarget{sensors-for-evaluation}{%
\section{Sensors for Evaluation}\label{sensors-for-evaluation}}

The developed framework is directly compatible with a variety of
magnetic field sensors without modifications, including those listed in
the table \ref{Implemented_digital_magnetic_field_sensors.csv}.

\begin{longtable}[]{@{}lll@{}}
\caption{Digital magnetic field sensors characterised for evaluation
\label{Digital_magnetic_field_sensors_characterised_for_evaluation.csv}}\tabularnewline
\toprule
& TLV493D-A1B6 & MMC5603NJ\tabularnewline
\midrule
\endfirsthead
\toprule
& TLV493D-A1B6 & MMC5603NJ\tabularnewline
\midrule
\endhead
Readout-Axis & 3D & 3D\tabularnewline
Temperature-Sensor & yes & yes\tabularnewline
Resolution {[}uT{]} & 98 & 0.007\tabularnewline
Range {[}mT{]} & ±130.0 & ±3\tabularnewline
Background-Noise {[}uT{]} & 100 & 0.2\tabularnewline
\bottomrule
\end{longtable}

For this evaluation, the sensors listed in the table
\ref{Digital_magnetic_field_sensors_characterised_for_evaluation.csv}
were used for sensor characterisation. The additional column for the
\emph{Background-Noise} is taken from the respective data sheets of the
sensors and will be verified in the later \emph{Background-Noise}
measurement \ref{sensor-characterisation-background-noise}.

This selection is made for the following reasons:

The developed framework is directly compatible with a variety of
magnetic field sensors without modifications, including those listed in
the table. For this evaluation, the sensors listed in the table were
used for sensor characterisation. This selection is made for the
following reasons:

\begin{itemize}
\tightlist
\item
  Availability of ready to use development boards
\item
  Specifications correspond to the application areas
\item
  Availability for testing
\end{itemize}

It is possible to carry out the sensor characterisation shown here for
other compatible sensors using the same procedure. Pre-configured
measurements \ref{command-line-interface} and analysis pipelines
\ref{programmable-data-processing-pipeline} are available for this
purpose are packaged with library.

\hypertarget{evaluation-sensor-setup}{%
\section{Evaluation Sensor Setup}\label{evaluation-sensor-setup}}

\begin{figure}
\centering
\includegraphics{./generated_images/border_Sensor_evaluation_plattform_with_TLV493D_and MMC5603_sensors_placed_with_thermal_conductive_glue_on_an_aluminium_baseplate.png}
\caption{Sensor evaluation plattform with TLV493D and MMC5603 sensors
placed with thermal conductive glue on an aluminium baseplate
\label{Sensor_evaluation_plattform_with_TLV493D_and MMC5603_sensors_placed_with_thermal_conductive_glue_on_an_aluminium_baseplate.png}}
\end{figure}

The sensor platform used here is an adapted version of the \emph{1D:
Single Sensor} \ref{d-single-sensor} sensor platform. The sensors to be
measured were fixed together on an aluminium plate with thermally
conductive adhesive. This compensates for thermal differences. This is
essential for the subsequent temperature deviation tests in order to
obtain comparable measurement results.

The setup is placed and pre-wired in the temperature chamber 24 hours
before the series of measurements were carried out. The insulated
housing of a \emph{Voron 2.4} 3D printer, which has a separately
controlled internal heating system, is used as the temperature chamber.
To verify the temperature, an additional thermometer \emph{VC-7055BT} is
placed on the base plate. A \emph{10mm} thick \emph{PTFE} insulation
plate is placed between the floor and the sensor base plate to prevent
direct and uneven heating of the base plate by the heated floor.

The graphic
\ref{Sensor_evaluation_plattform_with_TLV493D_and MMC5603_sensors_placed_with_thermal_conductive_glue_on_an_aluminium_baseplate.png}
shows this basic setup, the \emph{Raspberry Pi Pico} shown here is used
as the readout hardware, on which the \emph{Unified Sensor Firmware} is
running. With additional connected switch, its possible to isolate or
select a sensor or both sensors to be queried from the firmware.

A separate battery powered supply is used as low-noise power supply for
the sensors boards. An \emph{Raspberry Pi 4} is used as the host
computer, which is connected to the sensors via a \emph{Hailege
ADUM3160} \gls{usb} isolator and is placed outside the temperature
chamber.

For the software setup, \emph{MRPCli} \ref{command-line-interface} is
used to control and record the measurement series, with the functions of
the \emph{MRPDataVisualisation} \ref{mrpvisualisation} and
\emph{MRPAnalysis} \ref{mrpanalysis} packages from the library
\ref{software-readout-framework} being used for subsequent evaluation.
The recorded measurement series are automatically analysed using the
\emph{Programmable-Data Processing Pipeline}
\ref{programmable-data-processing-pipeline} and the results are
visualised.

\hypertarget{sensor-characterisation-background-noise}{%
\section{Sensor Characterisation:
Background-Noise}\label{sensor-characterisation-background-noise}}

\begin{figure}
\centering
\includegraphics{./generated_images/border_Sensor_evaluation_setup_for_noise_measurements.png}
\caption{Sensor evaluation setup for noise measurements
\label{Sensor_evaluation_setup_for_noise_measurements.png}}
\end{figure}

Measuring the noise in a magnetic field sensor requires a precise
procedure and a special measurement setup. First, the magnetic field
sensor is placed in a quiet environment to minimize external field
interference. The temperature chamber for all noise tests is set to
\(\mu_{trev}\)=21.0\(^{\circ}\) and the sensors are placed \emph{24}
hours before the measurement run in the final measurement configuration
inside of the chamber.

The procedure begins with the acquisition of the baseline by operating
the sensor without external magnetic fields. For this purpose, a sample
size of \emph{N=10000} measured values is recorded for the baseline
measurement.

The output signal of the sensor is then continuously measured and
recorded. It is important to carry out the measurement over a
sufficiently long period of time in order to record both short-term and
long-term fluctuations. For this purpose, \emph{N=2000} further measured
values were taken with a trigger and readout rate of one measurement per
second.

In order to quantify the noise, the \gls{sd} of the signal is
calculated. These parameters provide information about the variation of
the signal over time and therefore about the sensor background noise.

\begin{figure}
\centering
\includegraphics{./generated_images/border_Sensor_noise_evaluation_results_for_TLV493D_and_MMC5603NJ_with_N=2000_samples_and_no_averaging.png}
\caption{Sensor noise evaluation results for TLV493D and MMC5603NJ with
N=2000 samples and no averaging
\label{Sensor_noise_evaluation_results_for_TLV493D_and_MMC5603NJ_with_N=2000_samples_and_no_averaging.png}}
\end{figure}

The following figure
\ref{Sensor_noise_evaluation_results_for_TLV493D_and_MMC5603NJ_with_N=2000_samples_and_no_averaging.png}
shows the measured values of these sensors. In the following, these are
analysed for each sensor examined.

The following data is shown in the plots:

\begin{itemize}
\tightlist
\item
  Plot of the raw data of the sensor
\item
  Plot of the sensors internal temperature sensor
\item
  Background noise level with reference to the initial baseline
\item
  Histogram of the background noise level
\end{itemize}

The table \ref{Sensor_noise_evaluation_results.csv} lists the measured
values that were extracted from the measurement data of the sensors
\ref{Sensor_noise_evaluation_results_for_TLV493D_and_MMC5603NJ_with_N=2000_samples_and_no_averaging.png}.
These measured values are categorised below.

\begin{longtable}[]{@{}lllll@{}}
\caption{Sensor noise evaluation results
\label{Sensor_noise_evaluation_results.csv}}\tabularnewline
\toprule
Symbol & TLV493D & MMC5603NJ & Unit & Description\tabularnewline
\midrule
\endfirsthead
\toprule
Symbol & TLV493D & MMC5603NJ & Unit & Description\tabularnewline
\midrule
\endhead
\(\mu_{nl}\) & 0.79 & 0.00 & \% & Mean noise level\tabularnewline
\(\sigma_{nl}\) & 0.76 & 0.00 & \% & \gls{sd} noise level\tabularnewline
\(\mu_{rv}\) & 157.23 & -50.39 & \(\mu\)T & Mean sensor value
(Baseline)\tabularnewline
\(\sigma_{rv}\) & 172.00 & 0.20 & \(\mu\)T & \gls{sd} sensor
value\tabularnewline
\(\mu_{t}\) & 20.68 & 19.40 & \(^{\circ}\)C & Mean sensor
temperature\tabularnewline
\(\sigma_{t}\) & 0.53 & 0.00 & \(^{\circ}\)C & \gls{sd} sensor
temperature\tabularnewline
\(\mu_{trev}\) & 21.0 & 21.0 & \(^{\circ}\)C & Ambient
temperature\tabularnewline
\bottomrule
\end{longtable}

\hypertarget{sensor-temperature-analysis}{%
\subsection{Sensor Temperature
Analysis}\label{sensor-temperature-analysis}}

The temperature stability of the \emph{TLV493D} with a mean value of
20.68\(^{\circ}\)C and a \gls{sd} of \(\sigma_{t}\)=0.53\(^{\circ}\)C
indicates a consistent trend. This implies a constant tendency. The
close grouping of the measured values around the mean value indicates
good stability. The confidence interval is expected to be between
20.15\(^{\circ}\)C and 21.21\(^{\circ}\)C, which indicates a stable and
consistent temperature measurement. This result is more noisy compared
to the temperature stability of the \emph{MMC5603}.

Both sensors provide an offset to the measured chamber temperature
\(\mu_{trev}\).

With an additional measurement run with a different temperature setting
of 30.0\(^{\circ}\)C, the measured temperature deviations and offsets
remains constant.

The sensor internal temperature sensors of both tested sensors are
suitable to perform an ambient temperature compensation of measured
values and calibration of the sensor. This is considered in section
\emph{Temperature Sensitivity} \ref{temperature-sensitivity}.

It is recommended, however, to use a separate temperature sensor when
using the \emph{TLV493D} or to use a suitable averaging of the
temperature and measured values in order to perform temperature
compensation.

\hypertarget{raw-sensor-data-analysis}{%
\subsection{Raw Sensor Data Analysis}\label{raw-sensor-data-analysis}}

In the raw data plot, as well as its mean value \(\mu_{rv}\) of both
sensors, an offset can also be recognised, which is dependent on the
ambient conditions. For verification, a reference measurement of the
environment is carried out with the calibrated \emph{Voltcraft GM70}
Telsameter next to the sensor \gls{ic} This provides a \emph{real}
baseline value of \(\mu_{rev}\)=-21\(\mu\)T.

\hypertarget{sensor-characterisation-linearity}{%
\section{Sensor Characterisation:
Linearity}\label{sensor-characterisation-linearity}}

\begin{figure}
\centering
\includegraphics{./generated_images/border_Sensor_evaluation_setup_for_linearity_measurements.png}
\caption{Sensor evaluation setup for linearity measurements
\label{Sensor_evaluation_setup_for_linearity_measurements.png}}
\end{figure}

The sensor linearity of a magnetic field sensor describes the ability of
the sensor to provide a proportional linear response to changes in the
magnetic field without non-linear distortions. This means that the
output signals of the sensor vary directly proportional to the input
magnetic fields without deviations or distortions and is therefore an
important indicator for measurements of fields at different distances or
objects.

This is achieved here by means of an additional linear axis installed
above the sensor setup. A holder for an \emph{N45 12x12x12mm} magnet is
attached to the end effector of this axis, which can thus be moved at
different distances above the respective sensor \gls{ic}. The ambient
temperature is set to \(\mu_{trev}\)=21.0\(^{\circ}\) in the measurement
runs and thus corresponds to the same conditions as in the
\emph{Background-Noise}\ref{sensor-characterisation-background-noise}
setup.

The figure \ref{Sensor_evaluation_setup_for_linearity_measurements.png}
shows this updated measurement setup with the added components. To
control the linear axis, an additional motion controller of the type
\emph{SKR-Pico} placed outside the temperature chamber is required,
which can be controlled via a network interface.

\hypertarget{measurement-setup}{%
\subsection{Measurement Setup}\label{measurement-setup}}

After the sensor baseline has been determined with \emph{N=10000}
samples, the magnet is inserted into the holder. The linear axis is
moved by the user until it is maximally saturated. With the
\emph{TLV493D}, this corresponds to 250.000\(\mu\)T and below in low
resolution mode. \emph{MCC5603NJ} around 3000\(\mu\)T. This ensures that
the linearity is determined over the entire measuring range.

\hypertarget{measurement-run}{%
\subsection{Measurement Run}\label{measurement-run}}

The measurement run is then started using a script. The user defines the
path to be travelled by the linear axis. The complete range of
\emph{120mm} in \emph{1mm} steps is selected here. The following
automated process then runs as follows:

\begin{enumerate}
\def\labelenumi{\arabic{enumi}.}
\tightlist
\item
  Move the linear axis upwards by the selected distance
\item
  De-energise the axis motor
\item
  Create measurement using \emph{MRPCli} with N=2000 data points per
  reading
\item
  Export reading to filesystem with current linear axis position in
  meta-data and filename
\end{enumerate}

\hypertarget{linearity-analysis}{%
\subsection{Linearity Analysis}\label{linearity-analysis}}

After the measurements were exported, they were analysed for linearity
using \emph{MRPAnalysis} \ref{mrpanalysis} functions.

\begin{figure}
\centering
\includegraphics{./generated_images/border_Sensor_linearity_evaluation_results_for_TLV493D_and_MMC5603NJ.png}
\caption{Sensor linearity evaluation results for TLV493D and MMC5603NJ
\label{Sensor_linearity_evaluation_results_for_TLV493D_and_MMC5603NJ.png}}
\end{figure}

The figure
\ref{Sensor_linearity_evaluation_results_for_TLV493D_and_MMC5603NJ.png}
shows the visual representation of linearity as a plot. The distance
from the magnet to the sensor is plotted in \emph{mm} on the x-axis. The
measured value of the sensor is plotted on the y-axis. This is not
directly comparable for both plots, as the sensors have different
measuring ranges. To ensure comparability, the ideal curve is
determined. In order to be able to make quantifiable statements about
the measurement results, the mean and \gls{sd} deviation of these two
curves is determined.

For both sensors, the deviation is less than 1\% over the entire
resolution. With the \emph{MMC5603NJ} this is on average only
\emph{0.04\%}. With the \emph{TLV493D}, however, the \gls{sd} is
\emph{3.64\%}, for which the deviations at the end in particular (with
field strengths towards zero) are decisive. The previously performed
\emph{Background-Noise} \ref{sensor-characterisation-background-noise}
characterisation shows that the linearity deviation here is due to the
sensitivity of the sensor.

In general, the measured values correspond to the data sheet
specifications of both sensors, which specify a value of \emph{5\%}. It
is also possible to calculate these small deviations using curve fitting
methods. Suitable functions are implemented in the library.

\hypertarget{sensor-characterisation-temperature-sensitivity}{%
\section{Sensor Characterisation: Temperature
Sensitivity}\label{sensor-characterisation-temperature-sensitivity}}

The temperature sensitivity of magnetic field sensors describes how
sensitively the sensors output is sensitive to temperature changes. It
is important to ensure that temperature fluctuations do not affect
measurement accuracy. A low temperature sensitivity reduces errors due
to temperature changes. An accurate temperature sensitivity
characteristic is therefore crucial for subsequent precise magnetic
field measurements.

\begin{figure}
\centering
\includegraphics{./generated_images/border_Sensor_evaluation_setup_for_temperature_sensitivity_measurements.png}
\caption{Sensor evaluation setup for temperature sensitivity
measurements
\label{Sensor_evaluation_setup_for_temperature_sensitivity_measurements.png}}
\end{figure}

As the temperature sensor \emph{TLV493D} in particular produced very
different results in the previous measurements, an additional
temperature sensor is attached to the sensor circuit board for this
measurement. The figure
\ref{Sensor_evaluation_setup_for_temperature_sensitivity_measurements.png}
shows these modifications in detail. These changes makes it possible to
accurately determine the sensors \gls{ic} temperature. The temperature
measuring device \emph{VC-7055BT} can be analysed using a \gls{pc}
interface. The controller of the temperature chamber can also be
programmed via a \gls{pc} interface and a target temperature can be
specified.

With this setup, it is possible to automatically acquire measured values
from the sensors under controlled temperature conditions.

The same procedure is used as for the \emph{Linearity}
\ref{sensor-characterisation-linearity} measurement, except that instead
of moving the linear axis, the temperature of the temperature chamber is
systematically increased from 20\(^{\circ}\) to 50\(^{\circ}\). Between
each of these temperature changes, the system is given a waiting time of
30 minutes after reaching the target temperature.

The field of permanent magnets is very temperature-dependent and can
lose its magnetisation at higher temperatures (typically
\textgreater=80\(^{\circ}\) for non-high-quality type N magnets
\cite{magna-c}). The temperature range is selected so that it is
within a sufficient range for the application.

\hypertarget{temperature-sensitivity-analysis}{%
\subsection{Temperature Sensitivity
Analysis}\label{temperature-sensitivity-analysis}}

\begin{figure}
\centering
\includegraphics{./generated_images/border_Sensor_temperature_sensitivity_evaluation_results_for_TLV493D_and_MMC5603NJ.png}
\caption{Sensor temperature sensitivity evaluation results for TLV493D
and MMC5603NJ
\label{Sensor_temperature_sensitivity_evaluation_results_for_TLV493D_and_MMC5603NJ.png}}
\end{figure}

The figure
\ref{Sensor_temperature_sensitivity_evaluation_results_for_TLV493D_and_MMC5603NJ.png}
shows the measured data as a plot with the temperature measured by the
measuring devices on the X-axis and the sensor measured value on the
Y-axis. The ideal baseline is also shown as a red line. It can be seen
that the \emph{MMC5603NJ} shows a straight-line drop in the measured
field strength with increasing temperatures. However, this is very
constant with a value of \emph{-2 \(\mu\)T / \(^{\circ}\)C} and is
therefore predictable.

The graph of the \emph{TLV493d} is significantly steeper with a gradient
of \emph{-5.13 \(\mu\)T / \(^{\circ}\)C}, and it is also not as linear
as the \emph{MMC5603NJ}; there are clear jumps in the gradient. However,
the total change between the temperature regions of 175\(\mu\)T is less
than that of the \emph{MMC5603NJ} with 70\(\mu\)T, if the total
measurement range of the sensors is also taken into account.

In the evaluation (see figure
\ref{Sensor_temperature_sensitivity_evaluation_results_for_TLV493D_and_MMC5603NJ.png}),
a linear function was also calculated using curve fitting to determine
the temperature coefficients of the sensors. This makes it possible to
compensate these deviations based on the ambient temperature during the
software calibration.

\hypertarget{result-analysis-1}{%
\section{Result Analysis}\label{result-analysis-1}}

\begin{longtable}[]{@{}lllll@{}}
\caption{Overview of all characterised sensor properties
\label{Overview_of_all_characterised_sensor_properties.csv}}\tabularnewline
\toprule
Symbol & TLV493D & MMC5603NJ & Unit & Description\tabularnewline
\midrule
\endfirsthead
\toprule
Symbol & TLV493D & MMC5603NJ & Unit & Description\tabularnewline
\midrule
\endhead
\(\mu_{nl}\) & 0.79 & 0.00 & \% & Mean noise level\tabularnewline
\(\sigma_{nl}\) & 0.76 & 0.00 & \% & \gls{sd} noise level\tabularnewline
\(\mu_{rv}\) & 157.23 & -50.39 & \(\mu\)T & Mean sensor value
(Baseline)\tabularnewline
\(\sigma_{rv}\) & 172.00 & 0.20 & \(\mu\)T & \gls{sd} sensor
value\tabularnewline
\(\mu_{t}\) & 20.68 & 19.40 & \(^{\circ}\)C & Mean sensor
temperature\tabularnewline
\(\sigma_{t}\) & 0.53 & 0.00 & \(^{\circ}\)C & \gls{sd} sensor
temperature\tabularnewline
\(\mu_{trev}\) & 21.0 & 21.0 & \(^{\circ}\)C & Ambient
temperature\tabularnewline
\(\mu_{sl}\) & 0.25 & 0.04 & \% & Mean sensor linearity
deviation\tabularnewline
\(\sigma_{sl}\) & 3.64 & 0.56 & \% & \gls{sd} sensor
linearity\tabularnewline
\(\mu_{td}\) & -1.99 & -5.13 & \(\mu\)T / \(^{\circ}\)C & Mean sensor
temperature coefficients\tabularnewline
\bottomrule
\end{longtable}

Table \ref{Overview_of_all_characterised_sensor_properties.csv} shows a
summary of all recorded and analysed measured values of the two
characterised sensors \emph{TLV493D} and \emph{MMC5603NJ}. It can be
clearly seen that these differ significantly by a factor of \emph{x10}.

The \emph{TLV493D} performs seriously in the Senosr Noise measurement
and also performs worse than specified in the data sheet (98\(\mu\)T
instead of 175\(\mu\)T), but the large measuring range, which fulfils
the required specifications from chapter \emph{Research Question}
\ref{research-question-and-approach}, must be taken into account here,
which is not met by the \emph{MMC5603NJ}.

The \emph{MMC5603NJ} can be used directly without additional software
calibration for measuring permanent magnets. Even without additional
measurement averaging, very precise measurement results can be achieved,
which achieve a measurement accuracy of less than \emph{1000\gls{ppm}}.

However, due to the limited measuring range of \emph{±3mT}, direct
measurement of stronger magnets is not possible. The \emph{N45
12x12x12mm} magnets used in the application typically have a field
strength of around \emph{100mT} at a distance of \emph{10mm}.

The \emph{TLV493D}, on the other hand, is able to measure these
directly, but does not achieve the required accuracy due to strong noise
and steep temperature coefficients.

In the following chapter, recommendations for action are defined, which
were derived from the analysis results.

\hypertarget{recommendation-for-action}{%
\subsection{Recommendation for Action}\label{recommendation-for-action}}

\begin{itemize}
\tightlist
\item
  folgendes setup woird für eine vermessung nach der sensor auswertung
  emholen:
\item
  temperaturkomensation sowie linearitäts kompensation mit separaten
  temperatur sensor
\end{itemize}

are they suitable for <1000(+\gls{ppm}) ?

\begin{itemize}
\item
  https://onlinelibrary.wiley.com/doi/epdf/10.1002/mrm.28396
\item
  27cm / 50mT bore
\item
  nmr probes for better results
\item
  mit software calibrierung ist es möglich vermutlich möglich sowie
  mittelung ist es möglich auch mit dem tlv493d kanpp unter die 50uT zu
  kommen
\item
  mess durchläufe mit ist es mögloch von 175uT nach 70uT zu kommen mit
  avg=20 kommt man auf 45
\item
  sollte ein geeignerter ersatzsensor gefunden werden
\item
  beide sensoren sind jedoch durch chrakterisierung von
  permanentmagneten geeignet
\item
  beim tlv493d ist es möglich absolut werte zu erfassen.
\item
  beim mmc ist es möglich relativ magnete untereinander zu vergleichen
\end{itemize}

\hypertarget{conclusion-and-discussion}{%
\chapter{Conclusion and Discussion}\label{conclusion-and-discussion}}

\hypertarget{conclusion}{%
\section{Conclusion}\label{conclusion}}

This work describes the development of a universal Python library that
is used to efficiently process data from magnetic field sensors from
acquisition to analysis. In order to ensure a practical application and
to give users the opportunity to directly acquire their own magnetic
field data, cost-effective and easily reproducible hardware is also
developed.

The hardware is based on widely used magnetic field sensors and low-cost
microcontrollers, which enables an easily expandable and applicable
solution for measuring magnets with repeatable accuracy.

A particular focus is placed on expandability by the user.
Interchangeable modules allow the user to develop their own analysis
algorithms without having to redesign everything from scratch.

This extensibility and customisability is successfully demonstrated
during the evaluation. This underlines the performance of the developed
framework and shows that it is not only effective in the processing of
magnetic field sensor data, but also offers a flexible platform for the
implementation of user-specific analyses.

\hypertarget{outlook}{%
\section{Outlook}\label{outlook}}

A solid foundation has been built in this version of the framework,
which contains all the necessary functions and is ready for immediate
use. During development, particular emphasis is placed on comprehensive
documentation to make it easier to get started. Together with examples
for various usecases, a user can quickly evaluate the framework.

However, it should be noted that the framework has already been released
with its first stable version, but extensions and improvements are still
necessary. The stable version distributed via the package registry is
well suited for the intended purpose. All tests and evaluations took
place under normal conditions, especially for the developed hardware
sensors, as the \gls{mrp} library works successfully with the
measurement data.

On the software side, the focus is on integration for the support of
more professional measuring devices. Only in this way is it possible to
evaluate and improve the sensor hardware and quantify the measurement
results.

To summarise, it can be said that a solid software framework has been
created that can be used directly for the intended purpose. It provides
a suitable working foundation, but can be further developed by
integrating professional measurement devices to enable a more
comprehensive evaluation and improvement of the sensor hardware.








%%%%%%%%%%%%%%%%%%%%%%%%%%%%%%%%%%%%%%%%%%%%%%%%%%%%%%%%%%%%
% REFERENZEN
%%%%%%%%%%%%%%%%%%%%%%%%%%%%%%%%%%%%%%%%%%%%%%%%%%%%%%%%%%%%
%% Verschiedene Versionen, nach DIN 1505 zu zitieren
%\bibliographystyle{plaindin}
%\interlinepenalty=10000
%\bibliography{thesis_references}
\printbibliography
%%%%%%%%%%%%%%%%%%%%%%%%%%%%%%%%%%%%%%%%%%%%%%%%%%%%%%%%%%%%
% ABBILDUNGSVERZEICHNIS
%%%%%%%%%%%%%%%%%%%%%%%%%%%%%%%%%%%%%%%%%%%%%%%%%%%%%%%%%%%%
\pagebreak
\addcontentsline{toc}{chapter}{Figures} 
\listoffigures

%%%%%%%%%%%%%%%%%%%%%%%%%%%%%%%%%%%%%%%%%%%%%%%%%%%%%%%%%%%%
% ABBILDUNGSVERZEICHNIS
%%%%%%%%%%%%%%%%%%%%%%%%%%%%%%%%%%%%%%%%%%%%%%%%%%%%%%%%%%%%
\pagebreak
\listoftables


\pagebreak
\chapter*{List of Equations}
\addcontentsline{toc}{chapter}{Equations} 
\listofmyequations


%%%%%%%%%%%%%%%%%%%%%%%%%%%%%%%%%%%%%%%%%%%%%%%%%%%%%%%%%%%%
% CODEVERZEICHNIS
%%%%%%%%%%%%%%%%%%%%%%%%%%%%%%%%%%%%%%%%%%%%%%%%%%%%%%%%%%%%
\lstlistoflistings

%%%%%%%%%%%%%%%%%%%%%%%%%%%%%%%%%%%%%%%%%%%%%%%%%%%%%%%%%%%%
% ANHANG
%%%%%%%%%%%%%%%%%%%%%%%%%%%%%%%%%%%%%%%%%%%%%%%%%%%%%%%%%%%%
%\newpage
%\appendix % ANHANG EINLEITEN
%\hypertarget{attachments}{%
\chapter{Attachments}\label{attachments}}



\end{document}
