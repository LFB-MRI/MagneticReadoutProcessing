A large number of permanent magnets are used in the construction of
low-field MRI equipment on the basis of permanent magnets. The magnetic
properties of these magnets must be similar to a certain degree in order
to achieve a homogeneous B0 field, which is necessary for many setups.

Due to the complex manufacturing process of neodymium magnets, the
different properties, i.e.~the direction of magnetisation, can deviate
from each other. This affects the homogeneity of the field and the image
acquisition.

A passive shimming process is typically used to adjust the field
afterwards. This is complex and time-consuming and requires manual
corrections to the magnets used. To avoid this process, magnets can be
systematically measured in advance. Sensor accuracy and repeatability
Data acquisition, storage and subsequent analysis play an important role
in this methodology.

Several existing open source solutions implement individual parts, but
do not provide a complete data processing pipeline from acquisition to
analysis, and their data storage formats are not compatible with each
other.

For this use case, the MagneticReadoutProcessing library has been
created in this work. It implements all important aspects of
acquisition, storage and analysis, and each intermediate step can be
customised by the user without having to create everything from scratch,
thus encouraging exchange between different user groups.

Complete documentation, tutorials and tests enable users to use and
adapt the framework as quickly as possible. The framework was used to
characterise different magnets, which requires integrating magnetic
field sensors.

To evaluate the software framework and hardware, two digital sensors
were characterised and tested under the same conditions for noise,
temperature dependence and linearity. The result is that both sensors
can only be used under certain conditions, as the noise intensity is
higher than the specified 50 \(\mu\)T and the value ranges fluctuate too
much as a result. It is recommended to consider analogue sensors or NMR
probes in the future.
