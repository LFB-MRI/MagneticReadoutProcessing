In the construction of low-field MRI devices based on permanent magnets,
a large number of magnets are used. In order to realize a homogeneous B0
field with these magnets, which is necessary for many setups, the
magnetic properties of these magnets have to be as similar to a certain
degree. Due to the complex manufacturing process of neodymium magnets,
the different properties, the direction of magnetization, can deviate
from each other, which affects the homogeneity of the field. To adjust
the field afterwards, a passive shimming process is typically performed,
which is complex and time-consuming and requires manual corrections to
the magnets used. To avoid this process, magnets can be systematically
measured in advance. In this methodology, the recording, data storage
and subsequent evaluation of the data play an important role. Various
existing open-source solutions implement individual parts, but do not
provide a complete data processing pipeline from aqusation to analysis
and the data storage formats of these are not compatible to each other.
For this use case, the MagneticReadoutProcessing library was created,
which implements all major aspects of acquisition, storage, analysis,
and each intermediate step can be customized by the user without having
to create everything from scratch, favoring an exchange between
different user groups. Complete documentation, tutorials and tests
enable users to use and adapt the FRamework as quickly as possible. The
framework for the characterisation of different N45 neodymium magnets,
which requires the integration of magnetic field sensors, was used for
the evaluation.
