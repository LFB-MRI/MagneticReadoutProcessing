\documentclass[conference]{IEEEtran}
\IEEEoverridecommandlockouts
% The preceding line is only needed to identify funding in the first footnote. If that is unneeded, please comment it out.
\usepackage{cite}
\usepackage{amsmath,amssymb,amsfonts}
\usepackage{algorithmic}
\usepackage{graphicx}
\usepackage{textcomp}
\usepackage{xcolor}
\usepackage[ngerman]{babel}
\usepackage[acronym]{glossaries}
%\usepackage{setspace}
\makeglossaries
\newacronym{sg}{SG}{Serious Games}
\newacronym{xr}{XR}{Extended Reality}
\newacronym{vr}{VR}{Virtual Reality}
\newacronym{ar}{AR}{Augmented Reality}
\newacronym{cg}{CG}{Computer Grafik}
\newacronym{gbl}{GBL}{Game-Based Learning}


\def\BibTeX{{\rm B\kern-.05em{\sc i\kern-.025em b}\kern-.08em
    T\kern-.1667em\lower.7ex\hbox{E}\kern-.125emX}}
\begin{document}

\title{Implementation of an Open-Source modular mangetic field camera for usage in Low-Field MRI Systems\\

}


\author{\IEEEauthorblockN{1\textsuperscript{st} Marcel Werner Heinrich Friedrich Ochsendorf}
\IEEEauthorblockA{\textit{Fachhochschule Aachen} \\
\textit{Electrical Engineering and Information Technology}\\
Aachen, Deutschland \\
marcelochsendorf@alumni.fh-aachen.de}
%\and
}

\maketitle

%-----------------------------------------------------------------------------------------------------------------------------------------------
% \gls{uscf}
\begin{abstract}

\end{abstract}

\begin{IEEEkeywords}
Serious Gaming (\gls{sg}), \gls{vr}, Extended Reality (\gls{xr}), Computer Grafik (\gls{cg})
\end{IEEEkeywords}

\section{Introduction}


%-----------------------------------------------------------------------------------------------------------------------------------------------
%ERLÄUTERUNGEN

\section{Serious Gaming und extended Reality Technologien}

Um \gls{sg} in der Lehre der \gls{cg} evaluieren und einsetzen zu können, werden zuerst die Themenbereiche voneinander abgegrenzt und erläutert.

\subsection{Game-Based Learning (\gls{gbl})}
Mittels \gls{gbl} soll spielerisch Wissen vermittelt werden. Man unterscheidet in Board Game Simulations und Digital \gls{gbl}.
Erstere sind rein haptische und nicht digitale Spiele, wie beispielsweise Schach, welches strategisches Denken lehrt und fördert.
Letztere beinhalten die Wissensvermittlung mittels digitaler Möglichkeiten und ermöglichen genau wie Ersteres regelbasiert Aktionen und Interaktionen,
durch den digitalen Anteil ist die Varianz deutlich erhöht und die Komplexität wird gesteigert.
Zudem können zusätzliche Komponenten, wie Ton, Bild oder Reaktionen verwendet und die Aufmerksamkeit des Anwenders gesteuert werden.
Der Lernprozess soll durch das Spielen nicht nur kognitiv, sondern durch eine Vielzahl an Sinnesorganen angesprochen werden, sodass die Erinnerungsleistung der
synaptischen Repräsentation im Gehirn erhöht wird\cite{fabulagames}.
Mittels digitaler Medien kann zudem das Spiel dezentral und / oder vernetzt umgesetzt werden\cite{a3}.




\subsection{Serious Gaming (\gls{sg})}
Mit dem Ziel der Vermittlung von Inhalten und Kompetenzen sind \gls{sg}s, Instrumente zur Wissensvermittlung (s. Abbildung \ref{sgdef_fig}).
Wörtlich übersetzt bedeutet es ``Spiel mit ernsthaftem (Lern-)Ziel''.
Unterhaltung ist nur ein untergeordnetes Ziel dieser Spiele, jedoch nicht unbedeutend, da der Anwender im Spiel motiviert werden soll.
Einerseits kann erlerntes Wissen vertieft und wiederholt und andererseits neues Wissen erlernt werden.
Es ist zu beachten, dass \gls{sg} auch 2D und / oder nicht \gls{xr}-basierende Spiele beinhaltet.
Demnach können auch nicht digitale Spiele, wie Board Game Simulations, den \gls{sg} zugeordnet werden.
Nachfolgend wird den \gls{sg}s nur in Bezug auf \gls{xr}-basierte Spiele betrachtet.
Nicht digitale oder Spiele für 2D Ausgabegeräte werden hierbei nicht berücksichtigt.



\subsection{Extended Reality (\gls{xr})}
\gls{xr} (oder auch Cross Reality) ist eine Sammelbezeichnung für alle Kombinationen aus realen und virtuellen Umgebungen,
welche mittels Technologien erzeugt und über Mensch-Maschine-Interaktion gesteuert werden.
Dies schließt auch alle zukünftigen Technologien ein.
Mittels dieser Technologien und zugehörigen Computerprogrammen werden virtuelle, mehrdimensionale Umgebungen erzeugt (\gls{vr}) oder virtuelle Komponenten
der realen Umgebung ergänzt (\gls{ar}) (s. Abbildung \ref{xrvrar_fig}).
Die visuelle Wahrnehmung lässt Umgebungen real erfahren und spricht vermehrt sensorische Reize an.
Werden \gls{xr} als Basis für \gls{sg} verwendet, können diese Reize angesprochen und somit langfristig Informationen und detaillierteres Wissen vermittelt und
vom Gehirn gespeichert werden. Dies verbessert die Qualität des \gls{sg}, bedarf aber eines höheren Aufwands in Vorbereitung und Umsetzung als 2D oder nicht digitale Spiele.

\subsection{Lehrveranstaltung Computergrafik (\gls{cg})}
Im Allgemeinen beschreibt die Lehrveranstaltung \gls{cg} die Modellierung und Erzeugung virtueller Welten, sowie deren Abbildung bzw. Projektion auf 2D-Bilder. Es wird der Informatik zugeordnet.
Die Modellierung und Kombination von komplexen Geometrien und Formen und auch die Berechnung dieser Komponenten sind Teil der \gls{cg}.
Grafische Grundfunktionen, wie Farbdarstellungen oder simple Geometrien, sind Basis dieser Modellierungen.
Beschrieben werden alle diese Modellierungen durch diverse mathematische und computergestützte Algorithmen.
Zudem beinhaltet es die Darstellung von Objekten und Umgebungen in der Ebene, sowie die Abbildung von 3D Umgebungen auf 2D Medien.
Die Computergrafik ist Basis aller \gls{xr} Technologien.
Unabhängig davon, ob einzelne virtuelle Objekte dargestellt werden sollen oder vollständige Umgebungen erzeugt werden, müssen alle Bestandteile
mittels \gls{cg} berechnet und modelliert werden.

Nachfolgend wird ein Beispiel analysiert, welche die hochschuldidaktische Lehre der \gls{cg} mittels \gls{sg} und \gls{xr} unterstützen und fördern kann.

%-----------------------------------------------------------------------------------------------------------------------------------------------
%METHODIKEN

\section{Auswahl an verfügbaren Methoden der \gls{sg}-gestützten Lehre}

\subsection{2D \gls{sg} Lösung in der hochschuldidaktischen Lehre am Beispiel der Business Simulation \gls{sg}}



%-----------------------------------------------------------------------------------------------------------------------------------------------
%SHADER

\section{Embedded System Software}

\subsection{Automatic Sensor Numbering}




\section{Analysis Software Framework}

\subsection{Data analysis pipeline}




%-----------------------------------------------------------------------------------------------------------------------------------------------
%VERGLEICHE

\section{Evaluation}

%-----------------------------------------------------------------------------------------------------------------------------------------------
%FAZIT

\section{Conclusion}


%-----------------------------------------------------------------------------------------------------------------------------------------------
\begingroup

%\setstretch{1.05}
\begin{thebibliography}{00}
\bibitem{eduxrvrar2020usa} Thomas Alsop: "Leading applications of immersive technologies in the education sector in the next two years according to XR/AR/VR/MR industry
experts in the United States in 2020", in: Internetseite Statista, URL: https://www.statista.com/statistics/1185078/applications-immersive-technologies-xr-ar-vr-mr-education/, Abruf am 17.11.2021.

\vskip 0.05in
\bibitem{evallearningmixedreality2020} Y. M. Tang; K. M. Au; H. C. W. Lau; G. T. S. Ho, C. H. Wu; "Evaluating the effectiveness of learning design with mixed reality (MR) in higher education", Springer-Verlag London Ltd, 28.02.2020

\vskip 0.05in
\bibitem{a7} Linda Stege, Giel van Lankveld, Pieter Spronck; "Serious Games in Education", International Journal of Computer Science in Sport, 2011

\vskip 0.05in
\bibitem{fabulagames} Fabula Games, "LERNEN KANN SOOO BUNT SEIN", in: Internetseite Fabula Games, URL: https://fabula-games.de/, Abruf am 22.11.2021.

\vskip 0.05in
\bibitem{a3} Axel Jacob, Frank Teuteberg; "Game-Based Learning, Serious Games, Business Games und Gamification –Lernförderliche Anwendungsszenarien, gewonnene Erkenntnisse und Handlungsempfehlungen", Springer Fachmedien Wiesbaden GmbH, 2017

\vskip 0.05in
\bibitem{grundlagencg2001} Thomas Strothotte, Stefan Schlechtweg; "Grundlagen der Computergraphik", 2001

\vskip 0.05in
\bibitem{shadertool} Stefan Kraus, Timo Armbruster; "ShaderTool", in: Internetseite Steam, URL: https://store.steampowered.com/app/314720/ShaderTool/, Abruf 02.12.2021

\vskip 0.05in
\bibitem{a9} IBM Institute for Business Value; "Extended Reality (XR) can boost workforce efficiency and resilience by reimagining how work is done.", in: Internetseite IBM Institute for Business Value, URL: https://www.ibm.com/thought-leadership/institute-business-value/report/ar-vr-workplace, Abruf am 22.11.2021.

\vskip 0.05in
\bibitem{a10} Joseph F. Frederick; "Defining Serious Games", in: Internetseite flowleadership, URL: https://flowleadership.org/serious-games/, Abruf am 22.11.2021.

\vskip 0.05in
\bibitem{blender} Blender Foundation; "Blender", in: Internetseite Blender, URL: https://wiki.blender.org/wiki/Reference/Release\_Notes/3.0, Abruf am 22.11.2021.

\vskip 0.05in
\bibitem{unity3d} Unity Technologies; "Unity VR", in: Internetseite Unity Technologies, URL: https://unity.com/de/case-study\#games-vr, Abruf am 22.11.2021.

\vskip 0.05in
\bibitem{a11} TOPSIM; "TOPSIM-General Management-Teilnehmerhandbuch", in: Internetseite TOPSIM, URL: https://www.css.msm.uni-due.de/fileadmin/Dateien/MSM/Topsim-Gr8/01\_General\_Management\_Teilnehmerhandbuch\_STD\_15\_3.pdf, Abruf am 22.11.2021.

\vskip 0.05in
\bibitem{a14} Zehnan Feng, Robert Amor, Vicente Gonzales, Ruggiero Lovreglio; "Immersive Virtual Reality Serious Games for Evacuation Training and Research: A Systematic Literature Review", ResearchGate, 05.2018

\vskip 0.05in
\bibitem{a15} Rüppel, Schatz; "Designing a BIM-based serious game for fire safety evacuation simulations"; Advanced Engineering Informatics; 2011

\vskip 0.05in
\bibitem{googlecardboard} Google LLC; "Google Cardboard", in: Internetseite google.com, URL: https://arvr.google.com/intl/de\_de/cardboard/apps/, Abruf am 22.11.2021.

\vskip 0.05in
\bibitem{shadertoy} Inigo Quilez, Pol Jeremias; "Shadertoy", in: Internetseite shadertoy.com, URL: https://www.shadertoy.com, Abruf am 22.11.2021.

\vskip 0.05in
\bibitem{scholl} Ingrid Scholl, "Computergrafik" in: Internetseite Vorlesung Computergrafik von Fr. Prof. Scholl, Ingrid; URL: https://www.fh-aachen.de/menschen/scholl/lehre/computergrafik, Abruf am 01.12.2021.




\end{thebibliography}
\endgroup

\end{document}


